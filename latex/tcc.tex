\documentclass[12pt, a4paper, tocpage=plain]{abnt} % Fonte tamanho 12, papel A4, páginas do sumário sem o p.<número da página>

\usepackage[brazilian]{babel} % Gera datas e nomes em português com estilo brasileiro
\usepackage{hyperref} % Permite a criação de hyperlink no documento, como os links usados na referência
\usepackage[utf8]{inputenc} % Dá suporte para caracteres especiais como acentos e cedilha
\usepackage[T1]{fontenc}
\usepackage[alf]{abntcite} % Define o estilo de referência bibliográfica
\usepackage{graphicx} % Permite a utilização de imagens no documento
\usepackage[small]{caption} % Define as legendas das figuras com fontes menores do que o texto
\usepackage{pslatex} % Define que o formato da letra será Times New Roman
\usepackage{epigraph} % Permite a criação de epígrafes 
\usepackage{setspace} % Permite a definição de espaçamento entre linhas
\usepackage[top=3cm, left=3cm, right=2cm, bottom=2cm]{geometry} % Define as margens da folha

\setcounter{secnumdepth}{3} % Até três subsubsections numeradas
\setcounter{tocdepth}{3} % Até trẽs subsubsections numeradas

\setlength{\parindent}{1.25cm} % Define o recuo da primeira linha dos parágrafos para 1.25 cm

\usepackage{listings}
\usepackage{color}

\definecolor{dkgreen}{rgb}{0,0.6,0}
\definecolor{gray}{rgb}{0.5,0.5,0.5}
\definecolor{mauve}{rgb}{0.58,0,0.82}

\lstset{
  language=Python,% the language of the code
  numbers=left, %numeração de linhas à esquerda
  stepnumber=1,
  firstnumber=1,
  numberstyle=\tiny,
  extendedchars=true,
  frame=none,
  basicstyle=\footnotesize,
  stringstyle=\ttfamily,
  showstringspaces=false,
  captionpos=b,
  %language=Java, %deve ser definida na inclusão de cada trecho de código, 
  % pois podem existir linguagens diferentes em exemplos diferentes
  breaklines=true,
  breakautoindent=true,
  %estilos de comentário de uma e várias linhas
  keywordstyle=\color{blue},          % keyword style
  commentstyle=\color{dkgreen},       % comment style
  stringstyle=\color{mauve},         % string literal style
  frame=single,                   % adds a frame around the code
  extendedchars=\true,
  aboveskip=12pt,
  inputencoding=utf8,
}
\renewcommand{\lstlistingname}{Código}

\renewcommand{\ABNTchapterfont}{\bfseries} % Define a fonte do \chapter
\renewcommand{\ABNTchaptersize}{\large} % Define o tamanho da fonte do \chapter
\renewcommand{\ABNTsectionfontsize}{\large} % Define o tamanho da fonte da \section
\renewcommand{\ABNTsubsectionfontsize}{\large} % Define o tamanho da fonte do \subsection
\renewcommand{\ABNTsubsubsectionfontsize}{\large} % Define o tamanho da fonte do \subsubsection
\renewcommand{\ABNTbibliographyname}{REFERÊNCIAS BIBLIOGRÁFICAS} % Modifica o título gerado pelo \bibliographys

\begin{document} % Começo do TCC
\begin{titlepage}
 \begin{figure}[ht]
 \centering
 \scalebox{0.35}{\includegraphics{figuras/logo}}
 \end{figure}
 \begin{center}
   {\large BACHAREL EM SISTEMAS DE INFORMAÇÃO} \\ [3.5cm]
   {\large PRISCILA MANHÃES DA SILVA} \\ [4cm]
   {\large DESENVOLVIMENTO DE NOVAS FUNCIONALIDADES PARA A FERRAMENTA DE MANIPULAÇÃO DE ARQUIVOS EM NUVEM} \\
   \vfill
   {\large Campos dos Goytacazes/RJ} \\
   {\large 2012}
 \end{center}
\end{titlepage} % Cria a capa
\begin{titlepage}
 \begin{figure}[ht]
 \centering
 \scalebox{0.35}{\includegraphics{figuras/logo}}
 \end{figure}
 \begin{center}
   {\large BACHAREL EM SISTEMAS DE INFORMAÇÃO} \\ [3.5cm]
   {\large PRISCILA MANHÃES DA SILVA} \\ [4cm]
   {\large DESENVOLVIMENTO DE NOVAS FUNCIONALIDADES PARA A FERRAMENTA DE MANIPULAÇÃO DE ARQUIVOS EM NUVEM}\\ [2cm]
   \hspace{.45\textwidth} % posicionando a minipage
   \begin{minipage}{0.5\textwidth}
   \begin{espacosimples}
        Trabalho de conclusão de curso apresentado ao Instituto Federal Fluminense como requisito obrigatório para obtenção de grau em Bacharel de Sistemas de Informação.\\[1.5cm]
        Orientador: Prof. Rogério Atem Carvalho\\
        Co-orientador: Prof. Fernando Carvalho
    \end{espacosimples}
    \end{minipage}
   \vfill
   {\large Campos dos Goytacazes/RJ} \\
   {\large 2012}
 \end{center}
\end{titlepage}
 % Cria a folha de rosto
\begin{folhadeaprovacao}
    \setlength{\ABNTsignthickness}{0.4pt}
    \setlength{\ABNTsignwidth}{15cm}
    \setlength{\ABNTsignskip}{0.9cm}
    \begin{center}
        {\large PRISCILA MANHÃES DA SILVA} \\ [4cm]
        {\large DESENVOLVIMENTO DE NOVAS FUNCIONALIDADES PARA A FERRAMENTA DE MANIPULAÇÃO DE ARQUIVOS EM NUVEM}\\ [2cm]
        \hspace{.45\textwidth} % posicionando a minipage
        \begin{minipage}{0.5\textwidth}
        \begin{espacosimples}
        Trabalho de conclusão de curso apresentado ao Instituto Federal Fluminense como requisito obrigatório para obtenção de grau em Bacharel de Sistemas de Informação.\\\\
        \end{espacosimples}
        \end{minipage}
    \end{center}
    Aprovada em 22 de novembro de 2012 \\\\
    Banca avaliadora:
    \assinatura{Prof. Rogério Atem Carvalho (Orientador) \\ Doutor em Ciências de Engenharia / IFF Campus Campos \\ Instituto Federal de Educação, Ciência e Tecnologia Fluminense / Campus Campos Centro}
    \assinatura{Prof. Fernando Luiz de Carvalho Silva (Co-Orientador) \\ Mestre em Engenharia de Produção / UENF \\ Instituto Federal de Educação, Ciência e Tecnologia Fluminense / Campus Campos Centro}
    \assinatura{Prof. Fábio Duncan de Souza \\ Mestre em Pesquisa Operacional e Inteligência Computacional / UCAM Campos \\ Instituto Federal de Educação, Ciência e Tecnologia Fluminense / Campus Campos Centro}
\end{folhadeaprovacao}
 % Cria a folha de aprovação
\null
\vfill

{\normalsize \it \hfill Dedico este trabalho à minha mãe e amigos por todo apoio, compreensão \vspace*{4pt}

\hfill e contribuição que me prestaram.}  % Cria a folha de dedicatória
\begin{center}
\textbf{AGRADECIMENTOS}
\end{center}

Primeiramente agradeço a minha mãe que sempre esteve ao meu lado nestes longos anos. \\
Agredeço também a todos meus amigos e colegas de trabalho do NSI, que me apoiam, incentivam. \\
Bem como aos professores Rogério Atem e Fernando Carvalho que contribuiram muito para que esse projeto fosse realizado. % Cria a folha de agradecimentos
\null % o \vfill só funciona com o \null
\vfill
\epigraph{O computador não é mais o centro do mundo digital.}{Tim Cook} % Cria a epígrafe (onde se coloca um pensamento)
\begin{center}
\textbf{RESUMO}
\end{center}
\singlespacing

\noindent Este trabalho descreve sobre o desenvolvimento de novas funcionalidades para uma ferramenta livre de manipulação de documentos em larga escala, sendo o objetivo dessas funcionalidades permitir a ferramenta que se adapte a manipulação outros formatos de arquivos, de multimédia, por exemplo. Ainda é descrito no trabalho um pouco sobre as aplicações e/ou outras ferramentas que foram integradas a fim de tornar possível a existência destas funcionalidades. Assim é explicado também sobre como, através da parceria da empresa francesa Nexedi SA, com o Núcleo de Pesquisa em Sistemas de Informação(NSI-IFF), a ferramenta adquiriu a capacidade de manipular formatos correspondentes a arquivos de vídeo, áudio, imagens e PDF; cabendo a esta manipulação realizar a conversão e extração de dados destes arquivos. Encontram-se ainda, neste trabalho, conceitos básicos empregados para formação da ferramenta e sua integração com o núcleo linux, bem como conceitos da linguagem Python, a qual foi utilizada para seu desenvolvimento. \\

\noindent PALAVRAS-CHAVE:  Serviço Web, Escalabilidade, Software livre, Python
 % Resumo do trabalho
\begin{center}
\textbf{ABSTRACT}
\end{center}

\singlespacing

\noindent This work describes about the development of new functionalities to a free tool of document manipulation on a large scale, being this new functionalities responsible for allowing the tool to adapt in handling other file formats, such as multimedia ones, for example. It stills describes a little about the applications and/or other tools used for integrating its own with propose to handle with this new functionalities. Through this partnership between the French company Nexedi SA with the aid of the Nucleus of Research in Information Systems (NSI-IFF), was possible to implement the ability to manipulate shapes corresponding to video files, audio, images and PDF; being this manipulate as well responsible for converting, inserting and extracting data from this files. Still can find in this job, the structure of this tool, basic concepts used to create the tool and integrate it with the core Linux, as well as concepts of Python, which was used for its development. \\

\noindent KEYWORDS: Web Service, Free software ,Python, Cloud Computing 

 % Resumo em língua estrangeira
\renewcommand{\listfigurename}{LISTA DE FIGURAS} % Modifica o nome da lista de figuras
\listoffigures % Gera o índice de figuras
\renewcommand{\contentsname}{SUMÁRIO} % Modifica o nome do sumário
\tableofcontents % Gera o sumário
\onehalfspacing % Define o espaçamento de 1.5cm entre linhas
\chapter{INTRODUÇÃO}
\thispagestyle{empty}

Segundo \cite{TESLA}, a Internet é um produto descendente de um experimento militar americano que tentava formar uma rede que não fosse vulnerável ao ataque inimigo, e que desde seu processo de desmilitarização expandiu ao redor mundo, mesmo em continentes afastados e pouco populosos como a Antártica.  

Ela foi fundamental para criação do \textit{Cloud computing}, em português computação nas nuvens, que consiste em permitir ao usuário acesso a diversas aplicações e ao máximo de funcionalidades que esta nova forma de interação entre maquina e usuário possa disponibilizar, independente da plataforma em que o mesmo se encontre, tornando-se cada vez mais ampla a medida que o avanço tecnológico permite maior acesso a Internet.

Seguindo esta tendência, mesmo antes de sua popularização, o Núcleo de Pesquisa em Sistemas de Informação(NSI) já trabalha anos no desenvolvimento e melhoria do projeto Biblioteca Digital da RENAPI, o qual visa disponibilizar um acervo bibliográfico digital para disseminação de científico tecnológico produzido na rede de Instituições de Educação Profissional Científica e Tecnológica (EPCT).

Assim o NSI passou a utilizar, entre outras, a ferramenta \textit{OpenOffice.org Daemon}, a qual foi desenvolvida originalmente pela empresa francesa Nexedi SA. Essa ferramenta possuía por objetivo a conversão de documentos.

No entanto a partir do uso prolongado da mesma, foram identificados erros considerados em parte graves para uma aplicação desta extensão. Entre esses é possível citar perda de conexão, \textit{deadlock} no OpenOffice.org utilizado pela aplicação, estouro de memória entre outros.

Assim dada a necessidade de corrigir estes erros a fim de obter estabilidade, através da parceria do NSI com a Nexedi,foi realizada a analise desta ferramenta.

Entretanto a análise provou que era inviável corrigir tal ferramenta em função de novas demandas também percebidas.

Definiu-se assim uma nova meta, a proposta da criação de uma ferramenta que fosse capaz de manter as conversões desta, mas que tivesse seus principais erros corrigidos, e que ainda fosse capaz de extrair e manipular informações pertencentes aos documentos em questão.

Em 2010, a ferramenta Web Service OOOD 2.0 foi apresentada. Seguindo em parte a sigla da antecessora, esta correspondia as expectativas implícitas na ultima analise realizada, no entanto, dado que os objetivos anteriores foram alcançados, e com base no aprendizado envolvido neste desenvolvimento, novas metas foram traçadas.

Além de poder manipular documentos tornou-se desejável também que a ferramenta fosse capaz de manipular arquivos de imagem, vídeo, áudio e PDF.

\section{Objetivo}

Este trabalho tem como principal objetivo apresentar a ferramenta sucessora ao OOOD 2.0, com intuito de demonstrar suas atualizações e ampliações. O CloudOoo provou-se capaz de manipular outros formatos de arquivos, ainda que prematuramente não tenha mesma estabilidade, mas que caminha para isto.

\section{Estrutura do trabalho}

Este trabaho se divide em cinco capítulos a partir deste primeiro:

O segundo capítulo deste trabalho cita e explica sobre as principais ferramentas que permitiram ao CloudOoo alcançar os objetivos inicialmente traçados, e que estão diretamente envolvidas a este, demonstrando um pouco sobre cada uma delas.

No terceiro capítulo é apresentada a nova estrutura sobre a qual o CloudOoo vem sendo desenvolvido, e sobre sua necessidade para seu funcionament.

No quarto capítulo é apresentado um breve estudo de caso do desenvolvimento desta ferramenta, avaliando seu desenvolvimento; falando sobre aplicações que o utilizam e demonstrando sobre sua atual estabilidade com base em estudos anteriores.

Por fim no quinto capítulo são apresentadas as conclusões sobre este trabalho e de proposta futuras para a melhoria desta ferramenta.

\chapter{TECNOLOGIAS EMPREGADAS}
\thispagestyle{empty}

Para o desenvolvimento do CloudOoo foi essencial o uso de outras ferramentas as quais pudessem ser incorporadas neste e tornassem possível a criação de novas funcionalidades.
Este capítulo apresenta um conceito simplificado de cada uma dessas integrações, desde a linguagem principal empregada até as ferramentas utilizadas.


\section{Python}

Python é uma linguagem de programação poderosa, fácil de aprender de código aberto e livre. 

Ela possui estruturas eficientes e de alto nível de dados além de uma abordagem simples, entretanto eficaz, para programação orientada a objeto. Sua sintaxe elegante de tipagem dinâmica, juntamente com a sua natureza interpretável, tornam-na uma linguagem ideal para \textit{scripts} e desenvolvimento rápido de aplicações em muitas áreas, em diversas plataformas \cite{GUIDO}.

Por todas estas vantagens, hoje, Python é uma das linguagens mais utilizadas no mundo, assim muitas de suas bibliotecas e ferramentas são bem conhecidas. Abaixo serão descritas as ferramentas Buildout e Zope e a biblioteca Subprocess.


\subsection{Buildout}
\label{Buildout}

Buildout é uma ferramenta desenvolvida em Python, capaz de reproduzir o desenvolvimento de aplicações em um ambiente dedicado. Ele fornece suporte para criação dessas aplicações, especialmente para as desenvolvidas Python. 

Esta ferramenta se baseia na execução de procedimentos especificado em um arquivo de configuração conhecido por \textit{recipe}

A partir de um conjunto de características e variáveis de ambiente previamente definidas nesse \textit{recipe}, é gerada a estrutura da aplicação, sendo esta composta pelas dependências de sistema,por pacotes Python ou mesmo por outras aplicações.

Um único \textit{recipe} de uma aplicação pode conter definições de configuração, processos e ainda outras aplicações, que por sua vez podem ser organizadas com seu próprio \textit{recipe} \cite{BRANDOM}.

Levando em consideração o ideal do ambiente dedicado, também chamado de puro, a instalação do Buildout é extremamente simples podendo ser realizada a partir de um \textit{script} Python simples e de fácil acesso on-line, \cite{BOOTSTRAP}. Após executado, este \textit{script} gerará um segundo \textit{script} conhecido por ``bin/buildout''. 

Através do uso deste segundo \textit{script} é possível a execução das definições de cada \textit{recipe} a fim de obter a aplicação final. No exemplo \ref{xprompt} ocorre a construção da aplicação Xprompt.

{\singlespace
\begin{lstlisting}[caption=Exemplo de um \textit{script} Buildout,language=python,label={xprompt}]
[buildout]
develop = .
parts = 
  xprompt
  test

[xprompt]
recipe = zc.recipe.egg:scripts
eggs = xanalogica.tumbler
interpreter = xprompt

[test]
recipe = zc.recipe.testrunner
eggs = xanalogica.tumbler

\end{lstlisting}
}

O \textit{recipe} inicia-se com a declaração ``[Buildout]'', que é a única parte realmente obrigatória para identificá-lo como um \textit{recipe} do Buildout.

Caso a aplicação em questão esteja em desenvolvimento e/ou em fase de testes, utiliza-se a variável \textit{develop}, que deve informar a localização deste desenvolvimento.

Por \textit{parts} subentende-se o percurso do \textit{script}, isto é, a ordem de execução que o Buildout deve obedecer para instalação da aplicação independente da ordem que estejam descritas a aplicações ao longo do \textit{recipe}.

Cada \textit{part} possui uma descrição padrão que consiste de variáveis que envolvem cada componente responsável pela sua instalação. O \textit{eggs}, por exemplo, se trata da opção referente as dependências necessárias para execução desta parte.

O \textit{interpreter} é utilizado para gerar um interpretador Python contendo as dependências relacionadas em \textit{eggs}. Para utilizá-lo basta citá-lo e nomeá-lo. Neste caso o interpretador da aplicação será chamado ``xprompt''.

O tamanho de cada \textit{recipe} vária de acordo com a necessidade da aplicação. Neste caso com poucas especificações será possível gerar uma nova instância de uma aplicação Xprompt.

Ainda que incomum, é valido ressaltar que aplicações não necessariamente desenvolvidas em Python podem ter seu ambiente gerado pelo Buildout, com base na vasta disponibilidade de \textit{recipes} disponíveis. Isso geralmente é usado para suporte de dependências de sistema, em sua maioria desenvolvidas em C. O próprio Buildout será responsável por organizar e informar o caminho dessas aplicações, se necessário sobrescrevendo a variável de ambiente PATH.

\subsection{Subprocess}

O Subprocess é uma biblioteca Python que permite a criação de processos no sistema, similar a biblioteca Thread porém mais eficiente no quesito escalabilidade, pois permite que seus processos sejam divididos e utilizem diferentes processadores. 

Esta biblioteca foi criada na intenção de substituir outros módulos e funções obsoletas, e permitir também maior interação entre a aplicação e o sistema no qual se encontra.

{\singlespace
\begin{lstlisting}[caption=Exemplo de uso do Subprocess,language=python,label={codigo_subrocess}]
>>> import subprocess
>>> subprocess.Popen('echo "Hello World!"', shell=True)
Hello World!
\end{lstlisting}
}

No código \ref{codigo_subrocess} existe um pequeno exemplo de como o Subprocess pode ser utilizado.

Neste exemplo ele reproduz um acesso ao \textit{shell} do sistema a fim de demonstrar um \textit{Hello World} através do comando \textit{echo}, é possível observar o uso do \textit{shell} no final da segunda linha em que cita-se ``shell=True''.

Em aplicações que utilizam ambientes dedicados, como citado na sessão \ref{Buildout}, é preferencial que a variável \textit{shell} tenha valor \textit{false},a fim de usar as aplicações deste ambiente, entretanto este já é seu valor por padrão, assim não é preciso declará-lo, como é visto no código \ref{codigo_subprocess_convert}.

Nesses casos também é preferível o uso da variável ``env'' que representa variáveis de \textit{environment}, isto é, variáveis de ambiente. Ao substituir esta variável o Buildout direciona o processo a utilizar os binários que estão no seu ambiente próprio ao invés do ambiente original, que neste caso é o próprio sistema.

O processo de sobrescrever a variável ``env'' é similar ao passo de redefinir a variável \textit{PATH} do sistema.

{\singlespace
\begin{lstlisting}[caption=Exemplo de uso do Subprocess com \textit{PIPE} extraído do CloudOoo,language=python,label={codigo_subprocess_convert}]
command = ["convert", self.file, output]
stdout, stderr = Popen(command, 
                        stdout=PIPE,
                        stderr=PIPE,
                        env=self.environment).communicate()
\end{lstlisting}
}


No exemplo, código \ref{codigo_subprocess_convert}, também é utilizada a opção \textit{PIPE} que permite ao comando Popen retornar as mensagens de saída e erro respectivamente através das variáveis \textit{stdout} e \textit{stderr}, para que seu conteúdo possa ser após o uso do comando.


\subsection{Zope}
\label{zope}

Zope (Z Object Publishing Environment) é um serviço web livre, de código aberto, desenvolvido em Python \cite{ZOPE2}, em outras palavras, um \textit{framework} Python.

Zope já foi considerado o principal responsável pela popularização do Python. Não existiu ferramenta em Perl que o levasse as pessoas como o Zope levou o Python \cite{UDELL}. Entretanto o Zope se tornou muito extenso e complexo, requerendo uma alta de taxa de aprendizagem e assim sendo ofuscado pela ferramenta Django em questão de popularidade.tanto

Dado seu nível de complexidade e todas as extensões disponíveis, muitos continuam utilizando o Zope, indiferentes a sua queda de popularidade, fazendo com que suas extensões continuem surgindo, sendo mantidas em processo de atualização continua por seus mantenedores.


\subsubsection{Zope Interfaces}

Como citado na sessão \ref{zope}, diversas extensões Zope foram desenvolvidas com intuito de auxiliar na criação de outras aplicações Python, de forma a não deixá-las muito extensas ou pesadas. Entre estas a \textit{zope.interface} é um exemplo de extensão independente escrita em Python e mantida pela equipe Zope.

A \textit{zope.interface} foi criada com intuito de permitir a comunicação entre quaisquer componentes externos que possuíssem uma \textit{Application Programming Interface} (API). A idéia de criar uma interface para os componentes de uma aplicação é uma forma  elegante de resolver um antigo problema de tipagem dinâmica tratando as informações recebidas de forma genérica, a fim de renderizá-las para um tratamento mais específico.


\section{XML}

\textit{Extensible Markup Language} (XML) é uma linguagem em formato de texto simples para representação de informações sobre estruturas, sejam elas de documentos, dados, configurações, entre outros. 

Esta linguagem foi derivada de um formato antigo chamado SGML, e adaptada para ser mais flexível ao uso Web \cite{W3C-XML}.

Atualmente se tornou um dos formatos mais comuns para compartilhamento de informações em aplicações web, por ter uma escrita simples e corresponder a um padrão similar ao HTML, que por muitos anos já vem sendo utilizado.

Além disso esta linguagem dispõe de outras vantagens como: padrão de \textit{markup}, cujo o nome é detalhado de forma redundante; a descrição da estrutura, que de forma geral também permite uma compreensão bem literal, como se fossem textos; todas suas versões são correspondentes, isto é, mesmo que algumas delas procurem por \textit{markups} específicos, qualquer versão mais nova de XML funcionará em uma aplicação que utilizasse uma versão anterior ou mais antiga.


\section{Formato Aberto}

O formato aberto é uma especificação publicada para armazenar dados digitais, mantido geralmente por uma organização de padrões não proprietária, e livre de limitações legais no uso.

Um formato aberto deve ser implementável tanto em software proprietário quanto em livre, usando as licenças típicas de cada um. Em oposição a esta idéia o formato proprietário é controlado e defendido por interesses particulares de empresas proprietárias detentoras de seus direitos.

O objetivo principal dos formatos abertos é garantir o acesso a longo prazo aos dados sem incertezas atuais ou futuras no que diz respeito às direitos legais ou à especificação técnica. Um objetivo secundário dos formatos abertos é permitir a competição de mercado, em vez de deixar que o controle de um distribuidor sobre um formato proprietário iniba o uso de um produto de competição.

\subsection{Formatos abertos de Documentos}
\label{documentos}

O \textit{.txt} é o formato aberto mais comum para arquivos de texto por ser pequeno e na maioria dos casos dispor de vários programas de edição em qualquer plataforma operacional, entretanto não é considerado o formato ideal para documentos, uma vez que não possui opções de formatação; tais como itálico e negrito.

Em 01 de maio de 2005, surgiu o \textit{OpenDocument Format}(ODF), um conjunto de formatos para aplicações de escritório com o objetivo de padronizar os formatos abertos para documentos. Seu nome original era \textit{Open Document Format for Office Application}, uma iniciativa da \textit{Organization for the Advancement of Structured Information Standards}(OASIS), baseado em um XML criado por desenvolvedores do OpenOffice.org. Esta aplicação na época era uma das poucas capazes de utilizar sua estrutura.

Atualmente os formatos ODF existem a 7 anos e foram adotados por diversas aplicações. Mesmo em sotwares proprietário entre elas o Microsoft Office, mesmo sendo um software originalmente proprietário através de um \textit{pluggin} ele permite a edição e manipulação destes formatos.

Através da figura \ref{crescimento_odf}, adaptada de \cite{SILVA}, é possível acompanhar o crescimento do uso do ODF até 2010, ano em que fez 5 anos.

\begin{figure}[ht]
    \centering
    \scalebox{0.7}{\includegraphics{figuras/crescimento_odf}}
    \caption{Crescimento do ODF nos primeiros cinco anos.}
    \label{crescimento_odf}
\end{figure}

\begin{citacao}
O \textit{OpenDocument Format} ainda é uma incógnita à grande maioria dos usuários comuns, mas sua adoção cresce em várias partes do mundo, especialmente nos meios corporativos e governamentais. No Brasil, por exemplo, o ODF já conta inclusive com aprovação da Associação Brasileira de Normas Técnicas (ABNT), que aconteceu em 2008 (norma NBR ISO/IEC 26300)\cite{ALECRIM-ODF}.
\end{citacao}

\subsubsection{Formatos de documentos ODF}

O embora o mesmo padrão seja aplicado de forma geral em documentos ODF, esse padrão possui variações de extensões, no que diz respeito a documentos são elas:

\begin{itemize}
    \item{Para documentos de texto: .odt, .fodt;}
    \item{Para planilhas eletrônicas: .ods, .fods;}
    \item{Para apresentações: .odp, .fodp;}
    \item{Para bancos de dados: .odb;}
    \item{Para desenhos: .odg;}
    \item{Para fórmulas: .odf.}
\end{itemize}

E ainda existe um padrão a parte para modelos(\textit{templates}), são eles:

\begin{itemize}
    \item{Para documentos de texto: .ott;}
    \item{Para planilhas eletrônicas: .ots;}
    \item{Para apresentações: .otp;}
    \item{Para bancos de dados: .otb.}
\end{itemize}

Por guardar desde estrutura e dados textos até imagens presentes em seus arquivos a estrutura ODF é considerada um padrão comprimido assim como arquivo ZIP. É com base nesse padrão que diversas bibliotecas foram desenvolvidas para trabalhar com esses arquivos, entre elas a PyUno \ref{pyuno}.

\subsubsection{Estrutura de documentos ODF}

Como citado na sessão \ref{documentos} um documento ODF é uma estrutura de padrão aberto semelhante a arquivos comprimidos, como por exemplo arquivos ZIP.

Esta estrutura é composta principalmente por:

\begin{itemize}
    \item{mimetype: arquivo de linha única constituído pelo mimetype do documento;}
    \item{content.xml: arquivo que armazena o conteúdo criado pelo usuário do documento;}
    \item{meta.xml: arquivo responsável por armazenar os metadados do documento, ou seja, dados como autor, data de criação, data de modificação e outros;}
    \item{styles.xml: arquivo que contém os estilos do documento, tais como formatações de texto, parágrafos e outros;}
    \item{Pictures: pasta que armazena figuras existentes no documento.}
\end{itemize}

Existem ainda diversos arquivos e pastas que podem compor um documento ODF, mas que não são muito referidos em uso pratico.


\subsection{Formatos de Imagens}

No que diz respeito a imagens os formatos abertos tratam sobre patentes dos formatos, ou inclusa neles.

Assim em 1996 surgiu o formato \textit{Portable Network Graphics}(PNG) que tinha como principal objetivo substituir o formato GIF, portador de inúmeros algoritmos patenteados, que no entanto vieram a expirar em 2003.

Desde o princípio sua principal intenção foi sua utilização em qualquer aplicação sem necessidade de conflitos sobre patentes.

Além disso este formato permite comprimir imagens sem perda de qualidade e também a retirada do fundo de imagens através do canal alfa; possui suporte de milhões de cores, diferentemente do GIF, cujo suporte era de 256 cores; e ainda permite a criação de animações, cujas extensões podem variar em \textit{.mng} e \textit{.apng}, mas são igualmente livres.

Este formato é apoiado pela \textit{World Wide Web Consortium}(W3C), e se tornou um padrão internacional.

\subsubsection{Estrutura do PNG}

Um arquivo PNG consiste de uma assinatura PNG e seguido de vários blocos em serie \cite{PNG-BOOK}.

Sua assinatura equivale aos primeiros oito \textit{bytes}, consiste da serie “137 80 78 71 13 10 26 10”.

Cada bloco consiste de:

\begin{itemize}
    \item{length: inteiro correspondente ao numero de \textit{bytes} dos dados da imagem;}
    \item{chunk type: código equivalente ao tipo do bloco;}
    \item{chunk data: dados da imagem;}
    \item{Cyclic Redundancy Check (CRC): campo que contém o valor total de \textit{bytes} do bloco, o \textit{chunk type} e o \textit{chunk data}, mas sem o valor do \textit{length}.}
\end{itemize}

O inicio da serie de blocos deve conter um \textit{IHDR chunk} e o ultimo bloco deve conter um \textit{IEND chunk} como \textit{chunk type} sinalizando que representam o inicio e fim da serie respectivamente.

\subsubsection{Exif}

O \textit{Exchangeable Image File Format}(Exif) é um conjunto de metadados a respeito da imagem em questão, ou seja, são dados como autor, dia em que a foto foi tirada, câmera utilizada, entre outros listados conforme um padrão.

Todas essas informações ficam dentro da própria imagem, no entanto é preciso ter uma aplicação especifica para vê-lo, e em casos de informações mais abrangentes, as vezes é necessário possuir também uma aplicação para manipular a imagem e inserir nela metadados a respeito da imagem, aplicações estas como \ref{imagemagick}.

\subsection{Formatos de Áudio}

Os formatos abertos de áudio tratam sobre codecs disponíveis sem uma patente aplicada aos mesmos. Nesta categoria conta-se com os formatos Vorbis(.ogg) e \textit{Free lossless Áudio Codec}(.flac).

O .flac é um formato livre de áudio comprimível e sem perda de qualidade e dados durante a o processo de compreensão. Seu algoritmo permite que o arquivo reduza em até 60\% seu tamanho original, e também permite a manipulação de metadados.

O .ogg é um novo formato comprimível de áudio. 
É comparado de forma grosseira com outros formatos utilizados para guardar e reproduzir musicas, tais como MP3, VQF, AAC, e outros formatos de áudio digital. Mas é diferente de todos estes formatos porque é livre, aberto e sem patentes \cite{XIPH}.

\subsection{Formatos de Vídeo}

Assim como os arquivos de áudio, a licença dos formatos de vídeo tratam sobre patentes de codecs, e neste caso as extensões mais comuns são conhecidas por Theora(.ogv) e Matroska(.mkv).

Theora é de propósito geral, um codec de vídeo com perda de dados. É baseado no codec de vídeo VP3 produzido pela \textit{On2 Tecnologies} \cite{XIPH-THEORA}.

Matroska é o nome de uma iniciativa ousada para a criação de formatos universais de \textit{containers} de áudio e vídeo digitais integrados, também chamados de formato de vídeo \cite{WIKIPEDIA-MATROSKA}.

Assim o .mkv trata-se de um ``contentor'' de padrão aberto para vídeos, que pode conter vários dados de diferentes tipos de codificações.

\section{LibreOffice}

O LibreOffice é um pacote de software, ou seja, uma suíte para documentos de escritório, de produtividade compatível com a maioria dos pacotes semelhantes, e que esta disponível para várias plataformas. É um software de código aberto, livre para baixar, utilizar e distribuir \cite{LibreOffice}.

Seu inicio foi marcado em 2000, quando a Sun Microsystems liberou o código de seu produto StarOffice. Neste momento se chamava OpenOffice.Org. 

Em 2010 a comunidade que desenvolvia o projeto anunciou uma fundação independente, \textit{The Document Foundation}, a fim de cumprir com a independência explicita da carta anunciada ao inicio do projeto.

Assim no inicio de 2011 foi lançado o LibreOffice 3.3, que bem como o OpenOffice.org 2.0, já suportava a edição da suíte \textit{OpenDocument}.


\subsection{UNO}
\label{uno}

O projeto LibreOffice possuía uma característica muito útil e pouco utilizada que era a capacidade de integrar seu funcionamento com outros aplicativos, isto é, um componente foi disponibilizada a fim de que aplicações em outras linguagens que estivessem fora do projeto pudessem interagir com o mesmo. Esse componente é conhecido por UNO (Universal Network Objects), que por sua vez é composto por um modelo dos componentes do LibreOffice.

O UNO oferece interoperabilidade entre diferentes linguagens de programação, diferentes modelos de objetos, diferentes arquiteturas e processos, em uma rede local ou mesmo através da internet. Seus componentes podem ser implementados e acessados através de \textit{bindings} deste \cite{MINETTO-PYUNO}.

Atualmente existem \textit{bindings} para as linguagens C, C++, Java e Python. Desde a versão 1.1 o LibreOffice dispõe do PyUNO em suas instalações por padrão.

\subsubsection{PyUNO}
\label{pyuno}

O PyUNO representa uma ``ponte'' entre o LibreOffice e aplicações Python. Através dele é possível a manipulação do componente UNO, seção \ref{uno}, para utilizar praticamente todas funcionalidades disponíveis no LibreOffice por \textit{scripts} Python.

No entanto, segundo \citeonline{PYUNO}, essa ferramenta ainda não atingiu seu uso absoluto podendo conter diversos \textit{bugs}, erros, e assim dependendo em grande parte de colaboração por parte da comunidade que utiliza a mesma.

No código \ref{example_uno} é possível ver um exemplo pratico do uso do PyUNO em um código Python:

{\singlespace
\begin{lstlisting}[caption=Exemplo de uso do Uno,language=python,label={example_uno}]
import sys
import os
import uno

# Get the uno component context from the PyUNO runtime
uno_context = uno.getComponentContext()
# Create the UnoUrlResolver on the Python side.
url_resolver = "com.sun.star.bridge.UnoUrlResolver"
resolver = uno_context.ServiceManager.createInstanceWithContext(
  url_resolver,
  uno_context)
# Connect to the running OpenOffice.org and get its context.
uno_connection = resolver.resolve("uno:socket,host=%s,port=%s;urp;StarOffice.ComponentContext" % (host, port))
# Get the ServiceManager object
return uno_connection.ServiceManager
\end{lstlisting}
}

Neste exemplo extraído diretamente do CloudOoo utiliza-se o contexto do processo ativo do PyUNO para retorna um serviço de gerenciamento do UNO, isto é, uma conexão é estabelecida entre o principal serviço de gerenciamento do UNO e o PyUNO utilizado pelo CloudOoo para que este possa por fim controlar as ações deste serviço.


\section{ImageMagick}
\label{imagemagick}

ImageMagick é uma suíte de aplicações para criar, editar,compor ou converter imagens \textit{bitmap} \cite{IMAGEMAGICK-STUDIO}.
Na realidade o ImageMagick disponibiliza um conjunto de binários, compatíveis com varias plataformas, que no Linux são dados como comandos de níveis, separados para diversas funcionalidades.

Para \citeonline{TESLA} é uma ferramenta, originalmente criada por John Cristy, para visualizar e manipular imagens, que esta amplamente disponível na Internet.

As funcionalidades que podem ser consideradas mais comuns e utilizadas são \textit{convert} e \textit{identify}, essas funcionalidades podem respectivamente: converter imagens para outros formatos ou formatação, como invertido ou de girar em 180 graus; e identificar os metadados disponíveis na imagem, como autor, ou data que foi criada ou tirada.

Como as demais ferramentas citadas por este trabalho, é um \textit{software} livre, disponível para baixar e utilizar da forma desejada ao usuário.


\section{XPDF}

Xpdf é uma suíte de binários de código aberto para visualização e manipulação básica de FFMPEG arquivos \textit{Portable Document Format}(PDF), criada por Glyph e Cog \cite{GLYPH-COG}.

Além de permitir a visualização de arquivos PDF o XPDF é uma ferramenta que permite a extração de textos dentro de documentos PDF e a conversão dos mesmos para o formato \textit{postscript}, formato este especialmente composto de informações e desenvolvido originalmente pela \textit{Adobe System}.

Assim como a maioria das aplicações para Linux é utilizada através de comandos, neste caso o mais comum \textit{pdftotext} o qual captura o texto disponível no documento PDF e retorna este texto através de um arquivo de texto simples.


\subsection{Poppler}

O poppler é uma biblioteca para renderizar PDF baseada no xpdf 3.0 \cite{JOHNSON}.

É uma das bibliotecas de código livre mais utilizada pelos sistemas Linux para leitores PDF, seu desenvolvimento é idealizado pela FreeDesktop.Org.


\section{PDFTk}

Se um documento PDF é um trabalho eletrônico, então o PDFTk é utilizado de removedor de grampos, furador, pasta, entre outros utilitários de documentos. o PDFTk é uma ferramenta consideravelmente simples para fazer as tarefas diárias com documentos PDF \cite{STEWARD}.

Esta ferramenta esta sob licença GPL e utiliza bibliotecas que possuem suas próprias licenças de uso.

Na realidade de simples o PDFTk tem apenas sua forma de uso, pois realiza tarefas complicadas no contexto de documentos PDF. Sua confiabilidade é altamente elogiável dado que seu criador e principal mantenedor, Sid Steward, é também autor do livro “PDF Hacks”, que é considerado uma das referências do ramo.


\section{FFMPEG}

Para \citeonline{FFMPEG-SCALABLE}, a melhoria constante do uso do processamento de multimídia, requerida e obtida pela expansão multi funcional e acelerada dos equipamentos de hardware, requer também aplicações eficientes e escaláveis a medida que este processo avança. E partindo deste principio uma das ferramentas ideais para projetos escaláveis é o FFMPEG.

FFMPEG é um rápido conversor de vídeo e áudio que também consegue tratar informações de ambos. Ele é capaz de converter faixas arbitrarias de amostras e redimensionar vídeos através de filtros polifásicos de alta qualidade \cite{FFMPEG}.

Mais do que isso, o FFMPEG é uma suíte de aplicações via linha de comando capaz de converter, extrair e inserir metadados em arquivos de áudio e vídeo de simples entendimento, de fácil uso.


\section{SERVIÇOS WEB}

Segundo \citeonline{PIRNAU}, os serviços web representam a metodologia em que aplicações podem se comunicar através de mensagens assíncronas ou chamadas remotas. Assim pode se dizer que serviços web são aplicações acessáveis remotamente.

Toda empresa tem por objetivo prover serviços, sejam esses para própria empresa, ou para clientes que por sua podem ser outras empresas. A anos esses serviços têm sido automatizados, inicialmente aplicações \textit{desktop} eram criadas quando a empresa era pequena e possuía poucos computadores, ou a comunicação entre elas não era tão necessária.

Quando a rede passou a estar presente no dia a dia de forma geral essas aplicações foram evoluindo e buscando a comunicação entre as mesmas.

Este conceito na verdade trata da iniciativa por parte dessas empresas de retirar suas aplicações dos computadores e passá-las para potente servidores que disponibilizarão esta na internet, assim basta ter acesso a internet e é possível utilizar esta aplicação.


\section{XML-RPC}

É um protocolo para chamadas remotas que utiliza HTTP para transporte e XML para codificação. O XML-RPC foi desenhado para ser o mais simples o possível, enquanto permite que uma estrutura de dados complexa seja transmitida, processada e retornada \cite{XMLRPC}.

Foi originalmente criado por Dave Winer na \textit{UserLand Frontier}, e inspirado por outros dois protocolos, um deles também desenvolvido pelo próprio Dave Winer e outro que representava o começo do protocolo SOAP. Entretanto seu uso é bem mais simples de se utilizar e entender que o SOAP. Suas mensagens correspondem a uma requisição \textit{HTTP-POST}, enquanto composição do corpo da mensagem é escrita em XML, bem como a resposta que a requisição recebe.

Dada sua simplicidade e eficiência se tornou um protocolo muito popular e utilizado hoje nos dias atuais.


\section{WSGI}

O \textit{Web Service Gateway Interface}(WSGI) é uma interface de entrada de dados para serviços web. É também uma especificação para que serviços e aplicações web se comuniquem com outras aplicações web, embora possa ainda ser utilizada para outras funções. É um padrão Python, escrito sob a \textit{PEP} 333 \cite{WSGI}.

Esta interface foi escrita com o objetivo de fornecer uma forma relativamente simples e compreensiva de comunicação entre aplicações e servidores, ou pelo menos com a maioria das aplicações web em Python, e que ainda pudesse suportar componentes \textit{middleware}.


\subsection{Paster}

Paster se trata de um componente para serviços web, composto pelo Python \textit{Paste}, que segue o padrão da interface Python WSGI.

Possui dois níveis de linha de comando composto inicialmente por \textit{paster}, onde o segundo comando especifica o serviço desejado, como \textit{serve} no caso de estabelecer o servidor, seguindo como parâmetros o restante das informações necessárias para estabelecer o serviço desejado.

É considerado um dos mais simples servidores web para Python, no entanto pode ser utilizado assincronamente e manter uma escalabilidade considerável até 2000\textit{rps}.


\section{Git}

Git é um sistema de controle de versões distribuídas livre e de código aberto, projetado para lidar com qualquer projeto, desde o menor ao maior com rapidez e eficiência \cite{SOFTWARE-FREEDOM-CONSERVANCY}.

A historia do Git está muito relacionada a criação do Linux e de Linus Torvalds, seu criador, bem como com toda comunidade de desenvolvimento Linux. Durante anos a comunidade utilizou a ferramenta \textit{BitKeeper} para guardar a modificações do projeto.

Em 2005, após um problema com a proprietária deste, a comunidade decidiu criar sua própria ferramenta a partir da experiência com a anterior, houve um novo foco em: velocidade, \textit{design} simples, suporte para desenvolvimento paralelo, distribuição completa e a habilidade necessária para lidar com projetos grandes sem perda de velocidade e dados.

Assim, esse novo sistema de versionamento permite que qualquer repositório seja o centro do versionamento, deixando todo \textit{log} das modificações guardados nele sem que para isso precise de uma conexão a rede ou servidor geral.


\subsection{Git e Subversion}

Diferentemente do Git, o Subversion é um sistema de controle de versões centralizado, entretanto muito utilizado atualmente, principalmente por projetos livres.

Embora seja consideravelmente rápido, é extremamente desaconselhável para projetos grandes e principalmente desenvolvidos paralelamente.

\chapter{CloudOoo}
\thispagestyle{empty}

Em 2006, em função da necessidade da conversão de documentos a empresa francesa Nexedi SA, em parceria com Núcleo de Pesquisa em Sistemas de Informação (NSI), originou a construção da ferramenta \textit{OpenOffice.Org Daemon} (OOOD), com base no uso da ferramenta \textit{LibreOffice}, antiga suíte \textit{OpenOffice.Org}, para conversão de documentos, no entanto com o uso prolongado desta ferramenta em ambientes de produção, foram identificados erros relacionados com a aplicação e com a sua atuação com o LibreOffice. Entre eles estavam erros como perdas de requisições, \textit{deadlock} de processo, e \textit{memory leak}.

Assim foi ressaltada a necessidade de modificações a fim de que a aplicação se tornasse mais estável, além de adicionar novas funcionalidades, como de exportar documentos para outras extensões além de ODF, PDF por exemplo, e também manipular informações destes, conhecidas como \textit{metadados}.

O resultado da continuação do desenvolvimento foi a ferramenta \textit{Web Service OOOD 2.0}, posteriormente nomeada por CloudOoo, apresentada em 2010. Esta nova versão da ferramenta se provou bem mais estável no uso a longo prazo, embora fosse considerada mais lenta em processos individuais, devido aos tratamentos adicionados para os erros conhecidos, e suas modificações de novas funcionalidades.

Ao final do processo de melhoria da ferramenta, novas funcionalidades foram idealizadas para diferentes tipos de arquivos, tais como arquivos de áudio, vídeo, imagem e PDF. A princípio essas funcionalidades seriam as mesmas aplicadas a documentos, conversões e manipulações gerais, vindo a criação de funcionalidades especificas quando a ferramente fosse considerada estável.

Assim o CloudOoo apresentado neste capítulo, é um serviço Web livre e de código aberto, sob a licença LGPL, que foi desenvolvido através parceria da Nexedi e do NSI, na linguagem de programação \textit{python} e que utiliza o protocolo XML-RPC para troca de mensagens, que pode ser utilizado inteiramente ou em partes separadas.

\section{Estrutura}

Desde de sua estrutura anterior, o CloudOoo foi desenvolvido para trabalhar de forma genérica prevendo futuras mudanças. Sua estrutura contém as interfaces:

\begin{itemize}
    \item{IApplication: representa os métodos de controles as aplicações externas do servidor;}
    \item{IFile: representa métodos para manipulação dos arquivos recebidos;}
    \item{IOdfDocument: representa métodos de manipulação específica de documentos ODF;}
    \item{IHandler: representa os objetos que irão realizar a requisição emitida pelo cliente;}
    \item{IMonitor: representa métodos de controle e manuseio dos processos estabelecidos no servidor;}
    \item{IMimemapper: representa métodos utilizados para trabalhar com filtros;}
    \item{IFilter: representa métodos de tratamento de filtros;}
    \item{ILockable: representa os métodos de controles para região crítica do servidor;}
    \item{ITableGranulator: representa métodos para extrair tabelas de documentos;}
    \item{IImageGranulator: representa métodos para extrair imagens de documentos;}
    \item{ITextGranulator: representa os métodos para extrair o conteúdo de um documento em capítulos e parágrafos;}
    \item{IERP5Compability: representa os métodos de compatibilidade com o ERP5;}
    \item{IManager: representa os métodos utilizáveis entre cliente e servidor.}
\end{itemize}

As próximas subseções apresentam detalhadamente sobre cada interface e as principais classes a implementá-las.


\subsection{IApplication}
\label{iapplication}

Por possuir a opção instalação em um ambiente isolado onde tanto o CloudOoo, quanto suas ferramentas podem possuir instalação própria, a partir do uso do \textit{buildout}, foi preciso construir uma interface para controlar as funções dos processos utilizados pela aplicação, ou seja, uma classe que fosse capaz de carregar a configurações das aplicações, controlar a inicialização e finalização de cada processo, e que fosse capaz de verificar se continuavam rodando no sistema operacional, a partir de um identificador e/ou da porta que cada uma utilizasse.


\subsubsection{Application}
\label{application}

Esta classe implementa a interface IApplication e tem por objetivo controlar aplicações que estejam dentro do processo do CloudOoo, como o \textit{LibreOffice} que precisa estar iniciado para possibilitar a manipulação dos documentos.

Além dos métodos citados em \ref{iapplication} esta classe é capaz de apresentar erros ocorridos durante processos e também de retornar o \textit{pid} utilizado pela aplicação. E ainda um método responsável pelo endereço dessa aplicação, ou seja, onde esta estabelecida e em qual porta.


\subsection{IFile}
\label{ifile}

Esta interface propõe um contrato de tratamento de arquivos, a fim de assegurar uma resposta eficiente e consistente ao cliente. Nela são contidos métodos para que o conteúdo do arquivo seja guardado durante sua instanciação de forma no acontecimento de erros não previstos este co\textit{nteúdo pode ser recuperado, ou mesmo restaurado a forma original.


\subsubsection{File}
\label{file}

Com base na implementação da interface IFile, esta classe possui métodos para manter qualquer arquivo recebido do cliente no sistema apenas durante o uso do mesmo. 

Ao receber um arquivo ela escreve o mesmo no disco, podendo assim recuperar seus dados, e obter informações do mesmo, como seu caminho, por exemplo.

Após o uso deste arquivo, ela é instituída a removê-lo do sistema, bem como qualquer arquivo criado a partir deste, a fim de não esgotar o servidor com arquivos desnecessários.


\subsection{IOdfDocument}

Embora muito similar a interface IFile, seção \ref{ifile}, porém seu tratamento é especifico para documentos ODF, dada a complexidade de armazenamento e manipulação destes.


\subsubsection{Document}

Por ter sido inicialmente desenvolvido para documentos ODF a estrutura do CloudOoo é relativamente planejada para estes, que por possuírem uma estrutura complexa e compacta exigiram a criação de classes especificas. 

Como no caso desta classe, a qual tem seus métodos desenhados para trabalhar com estruturas XML. Que foi estuda e teve métodos desenvolvidos com base no tempo de estudo a respeito de manipulações de documentos ODF.

Assim, por exemplo, esta classe possui um método para capitalização do arquivo como um todo, e outro para obtenção apenas da parte referente ao seu conteúdo XML, isto é, o conteúdo de ``content.xml''.


\subsection{IHandler}
\label{ihandler}

Esta interface foi especificamente criada para estabelecer o contrato entre as aplicações externas utilizadas pelo servidor em função dos pedidos do cliente no que desrespeito a manipulação direta do arquivo, como no caso de conversões por exemplo, ou então extrações e inserções de \textit{metadados}.

Para as classe que implementam esta interface é recomendado o igual uso de objetos do tipo \textit{File}, seção \ref{file}, para manipulação destes arquivos.


\subsubsection{OOHandler}

Inicialmente nomeado em função do \textit{OpenOffice.Org}, este \textit{handler} é uma implementação específica de comunicação com o \textit{LibreOffice}, que trata de requisições especificas aos documentos a fim de manipulá-los.


\subsubsection{PDFHandler}

Por utilizar de duas ferramentas, o \textit{Poppler} e \textit{PDFtk}, esta classe foi nomeada em função do tipo de arquivo que é responsável, ou seja, arquivos PDF. 


\subsubsection{IMAGEMAGICKHandler}

No que se trata de ferramentas para imagens o ImageMagick é uma das melhores disponível a nível de comando, ela consegue inclusive manipular dados do tipo \textit{Exif}, que são \textit{metadados} referentes a imagens.

Assim com base na ferramenta que utiliza, esta classe é responsável pela conversão e extração de \textit{metadados} de arquivos de imagem, deseja-se ainda implementar uma funcionalidade de inserção destes, no entanto esta funcionalidade requer uma base de estudos maior.


\subsubsection{FFMPEGHandler}

O FFMPEG  é capaz de manipular arquivos de áudio e vídeo facilitando assim a criação de uma classe que pudesse ser responsável simultaneamente pela manipulação de ambos.

Assim como as demais classes que implementam o IHandler, esta classe é capaz de manipular os \textit{metadados} desses arquivos, bem como convertê-los para determinadas extensões do mesmo tipo.


\subsection{IMonitor}

Esta interface foi desenvolvida principalmente com base nos erros anteriormente obtidos com o uso do \textit{LibreOffice}, no entanto é importante para o sistema como um todo.

Seu uso estabelece controles sobre princípios básicos do sistema, como uso de memória, tempo de requisições, tempo de uso do processo, entre outros.


\subsubsection{Monitor}

Basicamente a classe Monitor funciona como uma simples implementação da interface \textit{IMonitor} a fim de estabelecer seus atributos principais, como por exemplo o tempo minimo entre a monitorações do sistema. 

As próximas subseções explicam detalhadamente sobre estes controles através da herança desta classe.


\subsubsection{MonitorMemory}
\label{monitormem}

Nas configurações do CloudOoo existem definições que podem ser modificadas de acordo com o sistema em que vai ser instalado, entre elas existe uma variável responsável pelo uso máximo de memória pelo \textit{LibreOffice}.

A partir dessa definição, dada em \textit{megabytes}, essa classe monitora o uso da memória do sistema, assim caso esta atinja seu limite máximo de uso, a aplicação é reinicia com intuito de limpar da memória mensagens trocadas e que não foram liberadas de uso da mesma, evitando assim o evento chamado \textit{memory leak}, o qual consiste no uso de toda memória do sistema.


\subsubsection{MonitorTimeout}
\label{monitortim}

Também entre as definições citadas na subseção \ref{monitormem}, existe uma variável responsável pelo tempo limite de execução de um determinado processo, caso este tempo seja excedido ocorre o que chamamos de \textit{timeout} e a aplicação é forçada a parar, sendo reiniciada posteriormente.

A utilidade desta limitação é dada pela ideia de que caso este tempo tenha excedido ocorreu algum erro durante o processo, provavelmente em função da resposta da aplicação, assim reiniciá-la pode resolvê-lo.


\subsubsection{MonitorSleepingTime}

Com intuito de poupar uso do sistema em momentos desnecessários esta classe foi criada para observar o momentos de inutilização da aplicação e após determinado tempo parar a mesma. Assim é possível economizar no uso de recursos, e disponibilizá-los para outras aplicações que possam vir a utilizar o mesmo.


\subsubsection{MonitorRequest}

A fim de conservar a estabilidade do CloudOoo, esta classe implementa um controle em função do valor máximo de requisições que podem ser respondidas por cada instancia da aplicação, bem como nas subseções \ref{monitormem} e \ref{monitortim} a variável de requisições limite é estabelecida nas configurações.

Caso o valor informado seja excedido a instancia é encerrada, em seguido uma nova instancia da mesma aplicação é iniciada.


\subsection{IMimemaper}

Em casos de aplicações como o \textit{LibreOffice} que representam uma suíte de menores utilitários é preciso reconhecer a extensão de arquivo específica para cada utilitário. 

De forma geral estas extensões são explicitas no nome do arquivo. Entretanto, caso de que esta não sejam explicitas, é preciso reconhecer o tipo de arquivo de alguma outra forma. 

Neste caso existem o \textit{mimetypes}, que são identificadores presentes no conteúdo do arquivo, que permitem decidir sua extensão. 

Esta interface propõe métodos ara lidar com a identificação desses \textit{mimetypes}.

Seu exemplo de uso é a classe \textit{Mimemapper} a qual será apresentada na próxima subseção.


\subsubsection{Mimemapper}
\label{mimemapper}

O CloudOoo possui seus próprio filtros para identificar e renderizar arquivos dentro de sua própria instancia, seção \ref{ihandler}.

No entanto, dada a necessidade de suas aplicações internas, tornou-se necessário identificá-los de forma a torná-los igualmente reconhecível dentro de cada aplicação específica.

No caso do \textit{LibreOffice} é possível, através do uso do UNO, extrair os \textit{mimetypes} e demais informações como filtros extraídos dos próprios arquivos.


\subsection{IFilter}

Citados na seção anterior, \ref{mimemapper}, os filtros podem ter demais propriedades que ao serem requisitadas precisam estar disponível de forma facilmente utilizável.

Com base neste principio de utilização, esta interface propõe um contrato para trabalhar com os filtros mais complexos da melhor forma possível e igualmente da forma mais simples.


\subsubsection{Filter}

Se comparada as outras classes e suas devidas interfaces, a classe \textit{Filter} pode ser considerada a que representa o métodos mais simples.

Ao ser iniciada ela guarda todos os dados dos filtro em diversos atributos que foram selecionados prevendo seu uso posterior na aplicação.


\subsection{ILockable}

Quando se possui-se um recurso compartilhado, isto é, um recurso representado por outra aplicação rodando em paralelo à aplicação principal, admiti-se também existir uma região crítica.

Esta visão ocorre devido ao fato que determinadas aplicações não conseguem atender a mais de um processo simultaneamente, assim é necessário controlar essa região crítica prevendo travá-la caso um processo já esteja em usando a mesma.

Com este intuito a interface \textit{ILockable] foi implementada.


\subsubsection{OpenOffice}

A classe \textit{OpenOffice} foi exclusivamente criada visando controlar a aplicação \textit{LibreOffice}, anteriormente conhecida por \textit{OpenOffice.Org}, a qual foi responsável por maior parte dos erros respondidos do CloudOoo. 

Esta classe estende o uso da interface \textit{IApplication}, através da classe descrita na seção \ref{application} e ainda implementa a interface anterior, \textit{ILockable}, para que assim seja possível controlar a aplicação, trancá-la e destrancá-la durante seu processo de uso.

A partir do método \textit{isLocked} é possível verificar se esta aplicação está trancada ou disponível para uso, evitando assim erros, como o \textit{deadlock} por exemplo.


\subsection{ITableGranulator, IImageGranulator, ITextGranulator}

Uma das principais funções desenvolvidas para documentos foi a \textit{granularização}, ela trata da extração de partes importantes de documentos que não sejam especificamente textos, como por exemplo tabelas e imagens. 

Dada a complexidade dessa tarefa foi necessário a implantação das novas interfaces, para que estes processos fossem realizados.

A interface \textit{ITableGranulator} é a interface responsável pelo processo de ``granularização'' específico de tabelas presentes nos documentos. Ela implementa funções respectivas a uma tabela comum, baseada nas linhas e colunas dessas tabelas.

Na interface \textit{IImageGranulator} ocorre a ``granularização'' das imagens presentes nestes documentos, que são extraídas em seu formato original, ou em \textit{PNG}.

Por fim através da interface \textit{ITextGranulator} é possível partir o documento em textos menores dividindo o mesmo a partir de seus parágrafos, ou mesmo em capítulos.

Apesar dessas funcionalidades tratarem diretamente da ``granularização'' de documentos, podemos subdividi-las em dois tipos de documentos: os documentos livres de formato ODF, onde o processo ocorre a partir estudos realizados anteriormente dos mesmos; e documentos PDF, que são igualmente livres, porém tem um tratamento complexo dada sua falta de detalhamento e especificações.


\subsubsection{OOGranulator}

Esta classe foi desenvolvida especialmente para tratar do documentos ODF.

Suas funcionalidades são escritas com base nos \textit{namespaces} encontrados por padrão no XML que compõe esses documentos.

Até o momento essa classe é a única a implementar as exatas três interfaces da subseção anterior.


\subsubsection{PDFGranulator}

Esta classe implementa apenas as interfaces \textit{ITableGranulator} e \textit{IImageGranulator}.

É específica para o tratamento de documentos PDF, que exigem a utilização de bibliotecas e aplicações para ``desmenbrá-lo'' e tratar o resultado desta tarefa.

Por não possuir tantas especificações quanto documentos ODF, tratá-lo para ``granularização'' tem sido um trabalho mais complicado e com resultados não tão desejáveis, esta classe se encontra em desenvolvimento e aprimoramento.


\subsection{IManager}

A IManager trabalha como a interface entre o cliente e servidor a fim de estabelecer um procotolo entre ambos.

Ela detém um padrão genérico para troca de informações entre diferentes tipos de arquivos, inclusive vídeos. 

E possui ainda os principais métodos para funcionalidades do CloudOoo, no que desrespeito a todos seus \textit{handlers}, ou seja, módulos para tratar diversos tipos de arquivos.


\subsection{IERP5Compability}

Esta interface estabelece as funcionalidades entre cliente e servidor especificamente para o uso da aplicação ERP5. 

Ela reescreve a chamada dos métodos da aplicação OOOD para utilizarem os novos métodos sem interferir nas requisições feitas pelo uso do ERP5, e outros possíveis clientes que utilizassem a antiga plataforma.


\subsubsection{Manager}

A classe Manager implementa as classes \textit{IManager, IERP5Compability, ITableGranulator, ITextGranulator} e \textit{IImageGranulator} com o proposito de interligar suas funcionalidades ao cliente que venha a requiri-las, em outras palavras esta classe é a principal responsável pela conectividade entre cliente, servidor, aplicação e funcionalidades.

Nela são absorvidos os dados importantes para o funcionamento do CloudOoo, como por exemplo a base de \textit{mimetypes}, os \textit{handlers} disponíveis neste, as pastas de trabalho deste, entre outros dados.

Assim a partir desta classe é possível iniciar o servidor do CloudOoo.

\chapter{ESTUDO DE CASO}
\thispagestyle{empty}

Este capítulo apresenta um estudo de caso do CloudOoo e sua implantação num ambiente dedicado.

Para este estudo foram escolhidas duas formas de instalação: a primeira a partir do uso do SlapOS, subseção \ref{slapos}; a segunda a partir do uso direto do \textit{buildout}, forma que é principalmente utilizada para desenvolvimento e para serviços do NSI.

Não existe praticamente diferença entre ambas as instalações, exceto pelo tempo que levam, e configuração final.

Além disso este capítulo também apresentará uma avaliação quanto a forma de desenvolvimento do CloudOoo em comparação a sua versão anterior e do uso de técnicas ágeis.


\section{Aplicações relacionadas ao CloudOoo}

Com seus quase três anos de produzido o CloudOoo já vem sendo utilizado a muito por três principais aplicações, sendo uma delas diretamente responsável por sua instalação, como citado na seção \ref{insslapos} .

A subseções a seguir apresentam sobre essas aplicações.


\subsection{Biblioteca Digital}

A Biblioteca Digital da Rede Nacional de Pesquisa e Inovação (RENAPI), é um projeto que visa disponibilizar um acervo bibliográfico digital para contribuir com a disseminação de material científico e tecnológico produzido na rede de Educação Profissional Científica e Tecnológica (EPCT), sendo esse material periódicos, teses, monografias, artigos entre outros. Assim esse disseminação visa colaborar na qualificação do material humano digitalizado e na disseminação de conhecimento.

Este projeto é um dos principais desenvolvidos no Núcleo de Pesquisa em Sistemas de Informação (NSI), que conta atualmente com 20 bolsistas e 20 pesquisadores.


\subsection{ERP5}

Um ERP é capaz de integrar processos e dados de uma organização, através de recursos tecnológicos que padronizam e automatizam os mesmos.

Muito embora seja um sistema propriamente dito, ele foca mais em processos do que em funcionalidades, ele mascará informações em funcionalidades transparentes \cite{PITRE-DESAI}.

No entanto, apesar de trazer muitas vantagens a organização, esse processo de automatização, de forma geral, é longo, de alto custo e complexidade, e até mesmo difícil de implementar.

De acordo com \citeonline{SMETS-CARVALHO}, esta situação que motivou a criação do ERP5, cujas ferramentas são de código aberto, permitindo que a organização modifique-o a fim de torná-lo mais flexível aos seus processos.

Ele também incorpora conceitos avançados como o de banco de dados orientados a objetos, um sistema de gerenciamento, de sincronização, variação, \textit{workflows}, e possibilita a implementação de \textit{Business Templates}.

Compreende-se assim o ERP5, como sendo um ERP de baixo custo de implantação e alta tecnologia para pequenas e médias empresas.


\subsection{SlapOS}
\label{slapos}

O SlapOS é um sistema operacional de código aberto para o uso de redes distribuídas em computação em nuvem, que se baseia em que tudo se trata de processos.

Para \citeonline{SMETS-CERIN-COURTEAUD} a computação em nuvem é dividida em três camadas, infraestrutura como serviço (IaaS), Plataforma como Serviço (PaaS) e \textit{Software} como Serviço (SaaS). Na IaaS esta o funcionamento virtual do computador e seu armazenamento, sob o ele é construído o Paas, que funciona como coração dos serviços, como servidor e bancos de dados. Finalmente sobre Paas estão as aplicações de uso do usuário.

Através de uma API unificada e simples, que requer poucos minutos para aprendizagem, este sistema combina computação em grade e o conceito de ERP para fornecer estas categorias previstas na compução em nuvem.

Dada sua abordagem unificada e arquitetura modular, ele tem sido usado como uma ferramenta de testes para \textit{benchmark} de bancos de dados NoSQL e para otimização do processo de alocação em nuvem.


\section{Ambiente de desenvolvimento}
\label{computadores}

O estudo apresentado neste trabalho foi desenvolvido no NSI e contou com a disponibilidade de um computador com processador Intel Core I5 CPU 650 3.20 x4, 4GB de memória RAM, 320GB de HD; bem como de um notebook com processador Intel Core 2 Duo CPU 2.13 Hz, 8GB de memória 50GB de HD disponíveis para o sistema; além de dois servidores do NSI um de processador Xeon CPU E5335 2.00 x8, 14 GB de memória e 20GB de HD e um segundo com processador Xeon CPU E5335 2.00 x4, 4GB de memória e 20GB de HD, em uma máquina virtual.

Os dois primeiros computadores contavam com o sistema operacional Ubuntu 12.04 Lt, enquanto os servidores utilizavam o sistema operacional Debian 6.0.


\section{Processo de Desenvolvimento}

Para \citeonline{PRESSMAN}, a garantia da qualidade de software esta diretamente ligada ao emprego de testes sobre o mesmo, onde esses testes representam a expectativa do usuário sobre a aplicação, e podem ser empregados de forma automatizada ou manual. Embora testes automatizados tendam a gastar mais tempo em relação a programação dos mesmos, sua cobertura sobre o produto garante menor porcentagem de erros quando executada a aplicação.

A técnica TDD (\textit{Test-Driven Development}), ou desenvolvimento orientado a testes, defende o desenvolvimento dos testes antes do desenvolvimento da parte funcional da aplicação, dado que os testes também fazem parte da mesma. A eficácia dela garanti que todas as expectativas da aplicação sejam testadas e portanto garantidas ao usuário.

No inicio da parceria do NSI e Nexedi no desenvolvimento do CloudOoo quase não se utilizavam testes, entretanto com seu crescimento como produto e apresentada a dificuldade em mantê-lo estável foi dado o começo ao desenvolvimento de teste, a técnica TDD ainda não era empregada neste primeiro momento em função da porcentagem ja desenvolvida do produto, atualmente porém vêm sendo empregada diretamente.

Segundo \citeonline{ASTELS}, projetos que utilizam TDD devem possuir uma suíte exaustiva de testes que por sua vez determinam o código que deve ser escrito. No entanto para uma aplicação de serviço web, existe certo grau de dificuldade de começar do zero apenas com testes, tendo por justificativa suas dependências como outras aplicações bem como a utilização de rede por este. 

Para a realização de testes unitários no CloudOoo foi preciso antes o desenvolvimento de \textit{scripts} que pudessem interligar todas as bibliotecas envolvidas para o funcionamento do mesmo. Também foram desenvolvido \textit{scripts} que ``ligam'' a aplicação e realizam testes na rede local, a fim de garantir que as respostas estejam corretas quando este serviço estive ativo e responder a conexões distantes.

Além dos testes, outra ferramenta que garante o desenvolvimento do CloudOoo é o uso de um sistema de controle de versão, como o \textit{subversion}, que era utilizado na versão 2.0, e que atualmente foi trocado pelo \textit{git} em função de vantagens como, por exemplo, o controle de versões distribuído. A importância desta ferramenta esta no controle do crescimento da aplicação, que por momentos contou com uma equipe de desenvolvimento sem contato constante e em diferentes períodos.

Assim por meio de ferramentas que podiam acompanhar as modificações da aplicação, bem como reverter determinadas alterações em casos de erros posteriores na mesma e dar um detalhamento de quando essas modificações ocorreram, foi possível seguir com este desenvolvimento.

Sob certo ponto de vista o uso destas práticas sugere sobre o desenvolvimento do CloudOoo severas semelhanças com os métodos ágeis, muito embora não exista a definição propriamente dita de nenhuma delas aplicadas diretamente sobre o projeto.

Em seu artigo \citeonline{SILVA-MONNERAT-CARVALHO}, comprovam o uso do TDD no CloudOoo, e de suas complicações de emprego, uma vez que inicialmente não existia total proficiência por parte dos desenvolvedores. Além disso foi verificado sobre o estabilidade do desenvolvimento do projeto, tomando por base a lista de mudanças armazenadas pelo controle de versão. Observou-se através deste artigo que embora no incio do projeto nenhum teste falhasse, conforme o mesmo foi acrescido de mudanças e novas funcionalidades testes que antes passavam passaram a apresentar erros antes não conhecidos, precisando assim de correções.

Estas implicações trazem ao meio de software uma compreensão muito similar ao do ramo de indústria, no que se desrespeita a aplicação de novas técnicas para garantir que quando um produto chega ao meio de produção possa continuar estável ao uso.


\section{Processo de instalação}

Para instalação via SlapOS, foi definido como pré-requisito a instalação do próprio SlapOS, sendo este dependente apenas da existência da instalação do Python, em qualquer versão uma vez que o SlapOS instala todos seus requisitos de funcionamento, inclusive o próprio Python. Variando de acordo com o sistema escolhido podem ser necessários demais pacotes de dependências de sistema.

Da mesma forma, para instalação via \textit{buildout}, forma um pouco mais simples, é necessário igualmente possuir a instalação do  Python e do Git como pré requisitos, uma vez que através deles e do uso da biblioteca \textit{bootstrap}, disponível em \cite{BOOTSTRAP}, é possível gerar o \textit{script bin/buildout}, bem como utilizá-lo para viabilizar a instalação do CloudOoo por seus arquivos de configuração, ou receitas.

Nas próximas subseções serão apresentadas as devidas instalações.


\subsection{Instalação via SlapOS}

Esta instalação foi realizada a partir da modificação do tutorial de instalação do ERP5 no SlapOS \cite{ERP5-SLAPOS}, o qual passa por modificações ocasionalmente em função das atualizações frequentes desta ferramenta.

Nestas subseções será apresentado o tutorial já modificado em sequência de instalação:


\subsubsection{Instalação do SlapOS}
\label{insslapos}

Está primeira subseção é a instalação padrão para qualquer ferramenta que utilize o SlapOS.

É aconselhado que seja realizada a instalação na pasta raiz do sistema, assim requerendo privilégios para instalação. É possível para o usuário optar por instalá-lo em sua pasta de usuário com algumas modificações no tutorial, entretanto em determinados momentos será podem ocorrer erros inesperados, e a exigência do uso de privilégio de qualquer forma.

Como é possível notar no código \ref{slapos-1}, cria-se uma pasta para instalação em \textit{/opt/slapos}, e também um arquivo de configuração \textit{buildout}.

Após a criação deste arquivo, através do uso do Python, é realizada a execução de um \textit{bootstrap} próprio da Nexedi.

{\singlespace
\begin{lstlisting}[caption=Primeira parte da instalação do SlapOS,language=bash,label={slapos-1}]
$ sudo mkdir /opt/slapos 

$ cd /opt/slapos
$ touch buildout.cfg
$ vi buildout.cfg

[buildout]
extends =
  http://git.erp5.org/gitweb/slapos.git/blob_plain/refs/tags/slapos-0.57:/component/slapos/buildout.cfg

$ sudo python -S -c 'import urllib2; print urllib2.urlopen("http://www.nexedi.org/static/\
packages/source/slapos.buildout/bootstrap-1.5.3-dev-SlapOS-002.py").read()' | python -S -

$ sudo bin/buildout -v
\end{lstlisting}
}

Com o termino do \textit{bootstrap} é executado o \textit{script bin/buildout}, que realiza a instalação dos componentes necessários ao SlapOS.


Após a instalação dos componentes e dependências é preciso configurar o mesmo, no código \ref{slapos-2}, existe um exemplo de arquivo de configuração e logo em seguida na figura \ref{slapos-rede}, existe a configuração de rede para uso do SlapOS.

É necessário que o \textit{slapproxy} seja iniciado e mantido rodando em \textit{background} sempre que esta máquina estiver ligada e/ou em uso, ele representa funcionalidades de conectividade do SlapOS \textit{master}.

{\singlespace
\begin{lstlisting}[caption=Arquivo de configuração do SlapOS,language=python,label={slapos-2}]
$ touch slapos.cfg

[slapos]
software_root = /opt/slapgrid
instance_root = /srv/slapgrid
master_url = http://127.0.0.1:5000/
computer_id = vifibnode

[slapformat]
computer_xml = /opt/slapos/slapos.xml
log_file = /opt/slapos/slapformat.log
partition_amount = 5
bridge_name = br1331
partition_base_name = slappart
user_base_name = slapuser
tap_base_name = slaptap
ipv4_local_network = 10.0.0.0/16

[slapproxy]
host = 127.0.0.1
port = 5000
database_uri = /opt/slapos/proxy.db

\end{lstlisting}
}

{\singlespace
\begin{lstlisting}[caption=Configurações de rede do SlapOS,language=sh,label={slapos-rede}]
$ sudo bin/slapproxy -vc /opt/slapos/slapos.cfg
$ sudo brctl addbr br1331
$ sudo ip l s dev br1331 up
$ sudo ip a l dev br1331 fd00::1/64
$ sudo vi /etc/network/interfaces

  auto br1331
  iface br1331 inet6 static
    adress fd00::1
    netmask 64
    bridge_ports none

$ sudo bin/slapformat -c /opt/slapos/slapos.cfg
$ sudo bin/slapgrid -c /opt/slapos/slapos.cfg

\end{lstlisting}
}

É possível reparar que o SlapOS é uma ferramenta adaptada ao IPV6, e necessariamente é preciso configurá-lo através da interface \textit{br1331}.

O \textit{script /etc/network/interfaces} do sistema também é configurado para habilitar as funcionalidades de IPV6, a fim de que não seja preciso iniciá-las sempre que reinicie o sistema.

Por fim nas duas ultimas linhas do código \ref{slapos-rede} são executados os \textit{scripts slapformat}  e \textit{slapgrid}, eles são responsáveis por registrar seu computador na nuvem e seus devidos \textit{slots} além de habilitar o cliente SlapOS em seu computador, respectivamente.


\subsubsection{Instalação do CloudOoo}

Após realizada a instalação do SlapOS, o passo de instalação de ferramentas através do mesmo é consideravelmente simples.

Para requerer a instalação do CloudOoo é necessário utilizar o \textit{slapconsole} citando o arquivo de configuração citado na subseção \ref{insslapos}, assim com o uso de uma instância do SlapOS(\textit{slap}), referenciada na terceira linha da figura \ref{requisicao-software}, informa-se a esta instância qual será sua função, isto é, seu produto específico. 

A partir do uso do \textit{slapgrid} a parte referente aos componentes deste produto será instalado de forma compilada pelo \textit{Buildout}.


{\singlespace
\begin{lstlisting}[caption=Requisição de instalação do CloudOoo no SlapOS,language=bash,label={requisicao-software}]
$ bin/slapconsole /opt/slapos/slapos.cfg

import slapos.slap.slap
slap = slapos.slap.slap()
slap.initializeConnection('http://127.0.0.1:5000/')

slap.registerSupply.supply(
    "http://git.rp5.org/gitweb/slapos.git/blob_plain/cloudooo:/software/cloudooo/software.cfg",
    computer_guid="vifibnode")
#teste que retorna id da instancia
computer_particion.getId()

$ sudo bin/slapgrid-sr -c /opt/slapos/slapos.cfg

\end{lstlisting}
}

Nesta referência em particular o ponteiro esta voltado para o \textit{branch} do CloudOoo, em função de suas configurações próprias que não foram incorporadas ao projeto do SlapOS.

Após a instalação dos componentes é necessário requerer uma instância, como em \ref{requisicao-instancia}.

Este processo também requer o uso do \textit{slapconsole}, de forma semelhante a instalação de componentes, no entanto é necessário fornecer um título a sua instância e é possível verificar se a mesma foi criada requirindo, por exemplo, seu \textit{id}.

{\singlespace
\begin{lstlisting}[caption=Requisição de uma instância do CloudOoo via SlapOS,language=bash,label={requisicao-instancia}]
$ bin/slapconsole /opt/slapos/slapos.cfg

import slapos.slap.slap
slap = slapos.slap.slap()
slap.initializeConnection('http://127.0.0.1:5000/')

computer_particion = slap.registerOpenOrder(
    "http://git.rp5.org/gitweb/slapos.git/blob_plain/cloudooo:/software/cloudooo/software.cfg",
    "Meu CloudOoo")
#teste que retorna id da instancia
computer_particion.getId()

$ sudo bin/slapgrid-cp -c /opt/slapos/slapos.cfg

\end{lstlisting}
}

Por fim é utilizado novamente o \textit{slapgrid}, mas desta vez com novos parâmetros para que a criação da instância seja finalizada.

Por se tratar de um serviço web em um sistema de nuvens, esta instalação compreende que após terminada deverá iniciar um servidor do CloudOoo pelas configurações estabelecidas no software.


\subsection{Instalação do CloudOoo.git}
\label{clougit}

Esta segunda instalação foi criada para ser mais flexível aos projetos do NSI que também utilizam o CloudOoo, bem como aos que tenham intenção de colaborar com esta ferramenta. 

Esta instalação esta disponível no Github do NSI, \cite{BUILD-CLOUDOOO}, e possui um \textit{README} com instruções para instalação. 

Além disso esta instalação esta orientada a fazer download do repositório igualmente disponível no Github do NSI, em \cite{NSI-CLOUDOOO}, que se mantém sempre na última versão de desenvolvimento, a mais atual.

Após o download do \cite{BUILD-CLOUDOOO}, são necessárias apenas duas instruções para esta instalação, \ref{buildout-cloudooo}:

{\singlespace
\begin{lstlisting}[caption=Modo de instalação do cloudOoo \cite{BUILD-CLOUDOOO},language=bash,label={buildout-cloudooo}]

$ python -S -c 'import urllib2; exec urllib2.urlopen("http://python-distribute.org/boostrap.py").read()'

$ bin/buildout -vv

\end{lstlisting}
}

Na primeira etapa utiliza-se o Python para fazer \textit{download} e rodar o \textit{script} do \textit{boostrap}, este irá fazer o download e disponibilizar o \textit{Buildout}. 

Ao termino desta fase, com o \textit{script buildout} acrescido de argumentos no intuito de torná-lo verboso e do arquivo padrão de configuração (\textit{buildout.cfg}), ocorre a instalação do CloudOoo e todos os componentes necessários, entre eles o LibreOffice, FFMPEG, ImageMagick e demais.

Diferentemente da instalação via SlapOS o servidor do CloudOoo não fica disponível automaticamente para uso, é preciso ainda utilizar o \textit{script bin/supervisord} para dar inicio ao servidor.

Ainda caso o usuário desejar é possível trocar a porta de funcionamento do servidor no arquivo \textit{cloudooo.cfg}, bem como outras configurações padrão, como o limite do uso de memória.


\section{Processos de requisições}

Para estabelecer conexão entre um cliente qualquer e o CloudOoo é necessário o uso de uma biblioteca compatível ao XMLRPC. No código \ref{conexao} foi realizado um exemplo de cada requisição básica disponível para todos \textit{handlers}, através de uma conexão estabelecida pela biblioteca \textit{xmlrpclib}:

{\singlespace
\begin{lstlisting}[caption=Exemplo prático de uso do CloudOoo,language=python,label={conexao}]
from base64 import encodestring
from xmlrpclib import Server

conexao = Server("http://localhost:23000")

arquivo = open("test.ogv").read()

novo-arquivo = conexao.convertFile(encodestring(arquivo), 'ogv', 'mpeg')

info = conexao.getFileMetadataItemList(encodestring(arquivo), 'ogv')

nova-info = conexao.updateFileMetadata(encodestring(arquivo), 'ogv', dict(Titulo="Arquivo teste"))

\end{lstlisting}
}

Na linha 6 do código a variável \textit{arquivo} recebe o conteúdo do arquivo \textit{test.ogv}.

Para conversão deste, na linha 8, é preciso codificá-lo para que durante sua passagem do cliente para o servidor esse arquivo não seja danificado, esta codificação realizada pelo uso da biblioteca \textit{base64}.

No momento da conversão o servidor vai receber os dados do cliente, vai decodificar o arquivo e identificá-lo como um arquivo de vídeo, sendo assim o mesmo será encaminhado para o FFMPEGHandler, que por sua vez irá convertê-lo para o formato \textit{mpeg}.

O FFMPEGHandler também será o responsável pelos métodos de \textit{getFileMetadataItemList}, linha 10, e \textit{updateFileMetadata}, linha 12, nos quais serão realizados respectivamente a extração e inserção de metadados no arquivo.

Na linha 12 é possível notar que para inserção de metadados, os novos dados a serem passados devem estar em um dicionário.

Observe também que nesta inserção de metadados, foi utilizado o \textit{nova-info}, como resposta da requisição de inserção de metadados. Isto porque esta requisição do CloudOoo retorna um arquivo com os dados inseridos no mesmo, e que já se encontra codificado para transporte.

Além destas requisições comuns a todos os \textit{handlers} existem também requisições que foram previamente implantadas apenas para documentos:

\begin{itemize}
    \item{getAllowedExtensionList: retorna extensões permitidas para determinado arquivo;}
    \item{getChapterItem: retorna o capítulo selecionado do documento;}
    \item{getChapterItemList: retorna todos capítulos do documento;}
    \item{getColumnItemList: retorna colunas da tabela selecionada;}
    \item{getImage: retorna imagem selecionada;}
    \item{getImageItemList: retorna lista de imagens em documento;}
    \item{getLineItemList: retorna lista de linhas da tabela selecionada;}
    \item{getParagraph: retorna parágrafos selecionado;}
    \item{getParagraphItemList: retorna lista de parágrafos de um documento;}
    \item{getTable: retorna tabela selecionada;}
    \item{getTableItemList: retorna lista de tabelas no documento;}
    \item{system.listMethod: retorna lista de métodos disponíveis no servidor.}
\end{itemize}

\section{Uso do CloudOoo na Biblioteca Digital}

Para disponibilizar seu acervo, a Biblioteca Digital tem como regra converter os arquivos recebidos seu formato livre compatível, para isso utiliza de requisições ao CloudOoo. Na figura \ref{submeter}, existe um exemplo de submissão de arquivos, do tipo Relatório, o qual passará pela conversão de DOC para ODT através de uma requisição ao CloudOoo, após a aprovação \ref{aprovar}:

\begin{figure}[ht]
    \centering
    \scalebox{0.4}{\includegraphics{figuras/submeter}}
    \caption{Pagina de submissão de documentos da Biblioteca Digital}
    \label{submeter}
\end{figure}

\begin{figure}[ht]
    \centering
    \scalebox{0.4}{\includegraphics{figuras/aprovar}}
    \caption{Pagina de aprovação de documentos da Biblioteca Digital}
    \label{aprovar}
\end{figure}

\newpage
Nas figuras não é implícito o uso da ferramenta, entretanto, uma vez que essa funcionalidade requer o uso de documentos em formato livre para realização de suas tarefas, é requerido o uso do CloudOoo para converter o arquivo em questão para um padrão livre e posteriormente realizar a funcionalidade de ``granularização''.

\begin{figure}[ht]
    \centering
    \scalebox{0.4}{\includegraphics{figuras/metadados}}
    \caption{Pagina de exibição de metadados de documentos da Biblioteca Digital}
    \label{metadados}
\end{figure}

\newpage
Na figura \ref{metadados} há uma demonstração da funcionalidade de extração de metadados, enquanto os grãos encontram-se nas figuras \ref{imagens}:

\begin{figure}[ht]
    \centering
    \scalebox{0.4}{\includegraphics{figuras/imagens}}
    \caption{Pagina de exibição de grãos do tipo imagem de documentos da Biblioteca Digital}
    \label{imagens}
\end{figure}

\newpage
Os grãos extraídos pelo processo de ``granularização'' são separados em imagens(\ref{imagens}) e tabelas, que não existem nesse caso, e dispostos para visualização do usuário.


\newpage
\section{Performance}

Para estabelecer diferença de performance em relação as versões anteriores do CloudOoo foram considerados dois testes.

O primeiro deles realizado em um trabalho anterior, por \citeonline{MONNERAT}, contava 3699 documentos no formato ODT que foram selecionados previamente, entre eles alguns documentos inválidos e outros em formatos desconhecidos. Por ser a conversão mais utilizada, foi decido que neste teste estes documentos seriam convertidos para PDF. 
Este teste foi realizado por meio do uso de um \textit{script}, neles além de prover a conversão também era provido o armazenamento do tempo de cada conversão, bem como o tempo total que todas as conversões levaram em cada versão do CloudOoo.

No resultado do primeiro teste notou-se que o OOOD 2.0 levará mais tempo para realizar a conversão de cada documento,entretanto por ser mais estável levou 10 horas para realizar o teste e apresentou 12 erros, enquanto o OOOD 1.0 apresentou 531 erros e levou 11 horas para realizar o teste.

No segundo teste foram escolhidos arquivos distintos, representado no código \ref{carga}, este teste selecionou 3 diferentes documentos DOC para serem convertidos para ODT; 3 documentos com PDF para serem convertidos para TXT, 1 arquivo de vídeo AVI de aproximadamente 100 MB que seria convertido para THEORA; um arquivo de áudio MP3 de aproximadamente 3 MB para ser convertido para VORBIS; e por fim uma imagem PNG de aproximadamente 30 KB para ser convertida para JPG.

\newpage
{\singlespace
\begin{lstlisting}[caption=\textit{Script} de teste,language=python,label=carga]
from random import randint
from base64 import encodestring, decodestring
from datetime import datetime
from xmlrpclib import ServerProxy
import magic


mime_decoder = magic.Magic(mime=True)

documents = ['ISSCWR6PaperTemplate.doc','SiteLeaderAppWinter2012.doc', 'Estonia.doc']
pdfs = ['Bankruptcies.pdf', 'WIA-IM_tfw_studentguide.pdf', '2012FinancialRpt.pdf']

type_convert_choices = { 0: [documents, "doc", "odt", 'application/vnd.oasis.opendocument.text'],
                  1: ["test.png", "png", "jpg", 'image/jpeg'],
                  2: ["test.mp3", "mp3", "ogg", 'application/ogg'],
                  3: ["test.avi", "avi", "ogv", 'application/ogg'],
                  4: [pdfs, "pdf", "txt", 'text/plain']
                }


proxy = ServerProxy("http://localhost:23000/RPC2")
id = 0
process = []

while True:

  name = 'test'
  file, source_format, destination_format, mime = type_convert_choices[randint(0,4)]
  data = ''
  if type(file) == str: 
    data = open(file).read()
  else:
    name =file[randint(0,2)]
    data =  open(name).read()
  done, mimetype = False, False
  try:
    file = proxy.convertFile(encodestring(data), source_format, destination_format)
    mimetype = mime_decoder.from_buffer(decodestring(file))
    if mimetype == mime:
      done = True
  except:
    done = "Error"
  process.append([id, name, source_format, mime, destination_format, mimetype, done, '\n'])
  id += 1

content = "  id     |          file       |     source         |   dest            |    done          \n" + str(process)
log = open('log.txt', 'w')
log.write(content)
log.close()

\end{lstlisting}
}

O objetivo deste segundo teste era focar nas conversões de arquivos para formatos abertos, que são mais utilizados nos projetos que o CloudOoo atende. 

Além disso a perspectiva era que um menor número de erros aparecesse em função dos inúmeros tratamento utilizados nesta ferramenta. Também haveria um menor número de conversões que a versão anterior em função do tamanho e tempo que arquivos de vídeo, imagem e áudio consomem comparados a documentos.

Para saber se as conversões foram realizadas com sucesso eram verificados os \textit{mimetypes} de cada conversão de acordo com o esperado.

Infelizmente não foi possível a utilização de diversos arquivos aleatoriamente como seria o caso de um ambiente de produção, em função da dificuldade para adquirir estes arquivos por meio da internet que tem sido cada vez mais restrita a downloads.

Neste segundo teste foram utilizados dois computadores diferentes de desenvolvimento, citados em \ref{computadores}, no notebook e no servidor de 4GB de memória.

No notebook a performance foi consideravelmente estável, em 10 horas o CloudOoo realizou 2034 conversões, entre elas 405 conversões foram de DOC para ODT e apenas 3 retornaram erro; 260 foram de PDF para TXT e nenhuma retornou erro; 307 foram de AVI para OGV Theora e nenhuma retornou erro; 384 foram de MP3 para OGG \textit{Vorbis}; e 678 foram de PNG para JPG sem retorno de erro.

Resultando assim em apenas 3 erros em 10 horas.

No servidor, entretanto o resultado foi menos promissor, o teste durou apenas 3 horas em função de um erro de estouro de memória no servidor, que detinha de aproximadamente 3 GB livres para o teste, mas que não se encontrava disposto apenas para o teste uma vez que estava em uso em tempo real. 

Durante essas 3 horas foram realizadas 611 conversões, entre elas 122 conversões foram de DOC para ODT e apenas 1 retornou erro; 77 foram de PDF para TXT e nenhuma retornou erro; 91 foram de AVI para OGV Theora e nenhuma retornou erro; 114 foram de MP3 para OGG Vorbis; e 207 foram de PNG para JPG sem retorno de erro.

De forma positiva o CloudOoo se manteve estável durante essas horas tanto no notebook quanto no servidor, pois mesmo com o estoura de memória no segundo, ele se manteve ativo. Entretanto com os resultados dos teste foi constatado que com as novas funcionalidades adicionadas precisam ser revisadas em função de escalabilidade e uso em tempo real a fim de que não volte a ocorrer estouros de memória entre outros possíveis erros. 

No que respeito aos erros de documentos, dada a falta de variedade, foram todos erros relativos aos documentos PDF com proteção contra conversão.

Na tabela \ref{clooo} mostrando os resultados desses testes:

\begin{table}[!t]
\caption{Comparação entre versões do CloudOoo por meio de testes.}
\label{clooo}
\begin{tabular}{|r|c|c|c|c|p{1.5cm}|p{2cm}|p{1.5cm}|}
\hline
Versões & Documentos & Imagens & Áudio & Vídeo & Total de erros & Total de Conversões & Total de horas \\
\hline
OOOD 1.0 & 3699 & 0 & 0 & 0 & 531 & 3699 & 11 \\
\hline
OOOD 2.0 & 3699 & 0 & 0 & 0 & 12 & 3699 & 10 \\
\hline
CloudOoo & 405+260 & 307 & 384 & 678 & 3 & 2034 & 10 \\
\hline
\end{tabular} 
\end{table}


\chapter{CONCLUSÕES}
\thispagestyle{empty}

\section{Objetivos alcançados}

Neste trabalho foi possível apresentar de forma simplificada porém descritiva os processos e tecnologias empregadas por trás do CloudOoo, uma aplicação livre e de código aberto, que se encontra disponível a acesso.

Também foi descrito sobre sua instalação e uso como ferramenta de conversão e manipulação de arquivos comuns aos tipos de documentos, imagens, vídeos, áudio e PDF.

Foi possível ainda demonstrar mais sobre as ferramentas que possibilitaram este trabalho, e explicitar sobre suas facilidades de instalação e uso, bem como da facilidade de associá-las e utilizar diversas funcionalidades das mesmas através da linguagem Python.

\section{Trabalhos futuros}

Apesar deste trabalho representar em grande parte a realização de objetivos propostos por um trabalho anterior com a aplicação CloudOoo, nota-se a necessidade de modificações futuras ao mesmo visando sua melhoria continua.

Entre elas visar maior estabilidade do projeto não só por meio dos testes implementados e seus acréscimos, bem como pela revisão da escrita do projeto em função das novas funcionalidades adquiridas, que se demonstraram poucos estáveis.

Umas vez que os tipos de arquivos atendidos pelo mesmo foram expandidos também há o interesse de estender funcionalidades mais complexas já aplicadas aos documentos, como por exemplo a ``granularização'' de arquivos de vídeo, que é um processo já disponível e implantado no projeto da Biblioteca Digital.

\chapter{CONCLUSÕES}
\thispagestyle{empty}

\section{Objetivos alcançados}

Neste trabalho foi possível apresentar as contribuições realizadas a ferramenta de conversão e manipulação de arquivos em nuvem, CloudOoo.

Sendo esta apresentação uma simples descrição sobre todos os processos por trás dessas novas funcionalidades, entre elas a conversão e manipulação de áudio e vídeo, e a granularização de documentos PDF.

Além disso este trabalho apresentou detalhadamente a nova estrutura do CloudOoo, descrevendo um pouco sobre o uso desta estrutura dentro do mesmo.

Foram apresentadas também as tecnologias empregadas no CloudOoo, sendo essas aplicações livres e de código aberto, que se encontram disponível para acesso; e ainda outras ferramentas similares a esta aplicação.

Houve também uma breve descrição sobre como instalar e usar esta aplicação como ferramenta de conversão e manipulação de arquivos comuns aos tipos de documentos, imagens, vídeos, áudio e PDF.

Por fim, foi possível afirmar sobre as modificações e melhorias dessas funcionalidades através de um estudo de caso em cima do uso real em ferramentas utilizadas por empresas ao redor do mundo; e também através da realização de testes de escalabilidade comparados entre si e entre testes anteriores.

\section{Trabalhos futuros}

Apesar deste trabalho representar em grande parte a realização de objetivos propostos por um trabalho anterior com a aplicação CloudOoo, nota-se a necessidade de modificações futuras visando a melhoria continua da ferramenta.

Entre essas melhorias, visar maior estabilidade do projeto não só por meio dos testes implementados e seus acréscimos, bem como pela revisão da escrita do projeto em função das novas funcionalidades adquiridas, que se apresentaram poucos estáveis podendo causar problemas com o uso excessivo de memória e do sistema como um todo.

Umas vez que os tipos de arquivos atendidos pelo mesmo foram expandidos também há o interesse de estender funcionalidades mais complexas já aplicadas aos documentos, como por exemplo a ``granularização'' de arquivos de vídeo, que é um processo já disponível e implantado no projeto da Biblioteca Digital.

Além dessas funcionalidades pretende-se que arquivos de multimídia também tenham seu dados tratados.

Por fim é de interesse do projeto que esta ferramenta seja capaz de trabalhar como um serviço RESTful para respostas mais simples e realizadas em diversas aplicações por uma interface JSON.

\bibliography{referencia} % Gera as referências bibliográficas
\end{document} % Fim do TCC
