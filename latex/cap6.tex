\chapter{CONCLUSÕES}
\thispagestyle{empty}

\section{Objetivos alcançados}

Neste trabalho foi possível apresentar de forma simplificada porém descritiva os processos e tecnologias empregadas no CloudOoo, uma aplicação livre e de código aberto, que se encontra disponível para acesso.

Também foi descrito sobre sua instalação e uso como ferramenta de conversão e manipulação de arquivos comuns aos tipos de documentos, imagens, vídeos, áudio e PDF; bem como sobre outras ferramentas similares a este projeto.

Foi possível ainda demonstrar mais sobre as ferramentas que possibilitaram este trabalho, e explicitar sobre suas facilidades de instalação e uso, bem como da facilidade de associá-las e utilizar diversas funcionalidades das mesmas através da linguagem Python.


\section{Trabalhos futuros}

Apesar deste trabalho representar em grande parte a realização de objetivos propostos por um trabalho anterior com a aplicação CloudOoo, nota-se a necessidade de modificações futuras visando a melhoria continua da ferramenta.

Entre essas melhorias, visar maior estabilidade do projeto não só por meio dos testes implementados e seus acréscimos, bem como pela revisão da escrita do projeto em função das novas funcionalidades adquiridas, que se apresentaram poucos estáveis.

Umas vez que os tipos de arquivos atendidos pelo mesmo foram expandidos também há o interesse de estender funcionalidades mais complexas já aplicadas aos documentos, como por exemplo a ``granularização'' de arquivos de vídeo, que é um processo já disponível e implantado no projeto da Biblioteca Digital.

Além dessas funcionalidades pretende-se que arquivos de multimédia também tenham seu dados tratados.

Por fim é de interesse do projeto que esta ferramenta seja capaz de trabalhar como um serviço RESTful para respostas mais simples e realizadas em diversas aplicações por uma interface JSON.
