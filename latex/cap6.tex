\chapter{CONCLUSÕES}
\thispagestyle{empty}

\section{Objetivos alcançados}

Neste trabalho foi possível apresentar as contribuições realizadas a ferramenta de conversão e manipulação de arquivos em nuvem, CloudOoo.

Sendo esta apresentação uma simples descrição sobre todos os processos por trás dessas novas funcionalidades, entre elas a conversão e manipulação de áudio e vídeo, e a granularização de documentos PDF.

Além disso este trabalho apresentou detalhadamente a nova estrutura do CloudOoo, descrevendo um pouco sobre o uso desta estrutura dentro do mesmo.

Foram apresentadas também as tecnologias empregadas no CloudOoo, sendo essas aplicações livres e de código aberto, que se encontram disponível para acesso; e ainda outras ferramentas similares a esta aplicação.

Houve também uma breve descrição sobre como instalar e usar esta aplicação como ferramenta de conversão e manipulação de arquivos comuns aos tipos de documentos, imagens, vídeos, áudio e PDF.

Por fim, foi possível afirmar sobre as modificações e melhorias dessas funcionalidades através de um estudo de caso em cima do uso real em ferramentas utilizadas por empresas ao redor do mundo; e também através da realização de testes de escalabilidade comparados entre si e entre testes anteriores.

\section{Trabalhos futuros}

Apesar deste trabalho representar em grande parte a realização de objetivos propostos por um trabalho anterior com a aplicação CloudOoo, nota-se a necessidade de modificações futuras visando a melhoria continua da ferramenta.

Entre essas melhorias, visar maior estabilidade do projeto não só por meio dos testes implementados e seus acréscimos, bem como pela revisão da escrita do projeto em função das novas funcionalidades adquiridas, que se apresentaram poucos estáveis podendo causar problemas com o uso excessivo de memória e do sistema como um todo.

Umas vez que os tipos de arquivos atendidos pelo mesmo foram expandidos também há o interesse de estender funcionalidades mais complexas já aplicadas aos documentos, como por exemplo a ``granularização'' de arquivos de vídeo, que é um processo já disponível e implantado no projeto da Biblioteca Digital.

Além dessas funcionalidades pretende-se que arquivos de multimídia também tenham seu dados tratados.

Por fim é de interesse do projeto que esta ferramenta seja capaz de trabalhar como um serviço RESTful para respostas mais simples e realizadas em diversas aplicações por uma interface JSON.
