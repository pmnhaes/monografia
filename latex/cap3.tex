\chapter{CLOUDOOO}
\thispagestyle{empty}

Em função da necessidade da conversão de documentos a empresa francesa Nexedi SA originou a construção da ferramenta OpenOffice.Org Daemon (OOOD), entretanto seu uso prolongado apresentou erros que precisavam de tratamento e ainda com novas demandas crescendo era necessário a realização de mudanças e até de uma nova ferramenta mais estável. Em parceria com o NSI, foi idealizado o desenvolvimento desta nova ferramenta.

O resultado da continuação do desenvolvimento foi a ferramenta OOOD 2.0, posteriormente nomeada por CloudOoo, apresentada em 2010. Esta nova versão da ferramenta se provou bem mais estável no uso a longo prazo, embora fosse considerada mais lenta em processos individuais, devido aos tratamentos adicionados para os erros conhecidos, e suas modificações de novas funcionalidades.

Ao final do processo de melhoria da ferramenta, novas funcionalidades foram idealizadas para diferentes tipos de arquivos, tais como arquivos de áudio, vídeo, imagem e PDF. A princípio essas funcionalidades seriam as mesmas aplicadas a documentos, conversões e manipulações gerais, vindo a criação de funcionalidades especificas quando a ferramenta fosse considerada estável.

Assim o CloudOoo apresentado neste capítulo, é um serviço Web livre e de código aberto, sob a licença LGPL, que foi desenvolvido através parceria da Nexedi e do NSI, na linguagem de programação Python e que utiliza o protocolo XML-RPC para troca de mensagens, que pode ser utilizado inteiramente ou em partes separadas.

\section{Estrutura}

Desde de sua estrutura anterior, o CloudOoo foi desenvolvido para trabalhar de forma genérica prevendo futuras mudanças. Sua estrutura contém as interfaces:

\begin{itemize}
    \item{IApplication: representa os métodos de controles as aplicações externas do servidor;}
    \item{IFile: representa métodos para manipulação dos arquivos recebidos;}
    \item{IOdfDocument: representa métodos de manipulação específica de documentos ODF;}
    \item{IHandler: representa os objetos que irão realizar a requisição emitida pelo cliente;}
    \item{IMonitor: representa métodos de controle e manuseio dos processos estabelecidos no servidor;}
    \item{IMimemapper: representa métodos utilizados para trabalhar com filtros;}
    \item{IFilter: representa métodos de tratamento de filtros;}
    \item{ILockable: representa os métodos de controles para região crítica do servidor;}
    \item{ITableGranulator: representa métodos para extrair tabelas de documentos;}
    \item{IImageGranulator: representa métodos para extrair imagens de documentos;}
    \item{ITextGranulator: representa os métodos para extrair o conteúdo de um documento em capítulos e parágrafos;}
    \item{IERP5Compability: representa os métodos de compatibilidade com o ERP5;}
    \item{IManager: representa os métodos utilizáveis entre cliente e servidor.}
\end{itemize}

As próximas subseções apresentam detalhadamente sobre cada interface e as principais classes a implementá-las.


\subsection{IApplication}
\label{iapplication}

Por possuir a opção instalação em um ambiente isolado onde tanto o CloudOoo, quanto suas ferramentas podem possuir instalação própria, a partir do uso do Buildout, foi preciso construir uma interface para controlar as funções dos processos utilizados pela aplicação, ou seja, uma classe que fosse capaz de carregar a configurações das aplicações, controlar a inicialização e finalização de cada processo, e que fosse capaz de verificar se continuavam rodando no sistema operacional, a partir de um identificador e/ou da porta que cada uma utilizasse.


\subsubsection{Application}
\label{application}

Esta classe implementa a interface IApplication e tem por objetivo controlar aplicações que estejam dentro do processo do CloudOoo, como o LibreOffice que precisa estar iniciado para possibilitar a manipulação dos documentos.

Além dos métodos citados em \ref{iapplication} esta classe é capaz de apresentar erros ocorridos durante processos e também de retornar o \textit{pid} utilizado pela aplicação. E ainda um método responsável pelo endereço dessa aplicação, ou seja, onde esta estabelecida e em qual porta.


\subsection{IFile}
\label{ifile}

Esta interface propõe um contrato de tratamento de arquivos, a fim de assegurar uma resposta eficiente e consistente ao cliente. Nela são contidos métodos para que o conteúdo do arquivo seja guardado durante sua instânciação de forma no acontecimento de erros não previstos este conteúdo pode ser recuperado, ou mesmo restaurado a forma original.


\subsubsection{File}
\label{file}

Com base na implementação da interface IFile, esta classe possui métodos para manter qualquer arquivo recebido do cliente no sistema apenas durante o uso do mesmo.

Ao receber um arquivo ela escreve o mesmo no disco, podendo assim recuperar seus dados, e obter informações do mesmo, como seu caminho, por exemplo.

Após o uso deste arquivo, ela é instituída a removê-lo do sistema, bem como qualquer arquivo criado a partir deste, a fim de não esgotar o servidor com arquivos desnecessários.


\subsection{IOdfDocument}

Embora muito similar a interface IFile, seção \ref{ifile}, porém seu tratamento é especifico para documentos ODF, dada a complexidade de armazenamento e manipulação destes.


\subsubsection{Document}

Por ter sido inicialmente desenvolvido para documentos ODF a estrutura do CloudOoo é relativamente planejada para estes, que por possuírem uma estrutura complexa e compacta exigiram a criação de classes especificas. 

Como no caso desta classe, a qual tem seus métodos desenhados para trabalhar com estruturas XML. Que foi estuda e teve métodos desenvolvidos com base no tempo de estudo a respeito de manipulações de documentos ODF.

Assim, por exemplo, esta classe possui um método para capitalização do arquivo como um todo, e outro para obtenção apenas da parte referente ao seu conteúdo XML, isto é, o conteúdo de ``content.xml''.


\subsection{IHandler}
\label{ihandler}

Esta interface foi especificamente criada para estabelecer o contrato entre as aplicações externas utilizadas pelo servidor em função dos pedidos do cliente no que desrespeito a manipulação direta do arquivo, como no caso de conversões por exemplo, ou então extrações e inserções de ``metadados''.

Para as classe que implementam esta interface é recomendado o igual uso de objetos do tipo File, seção \ref{file}, para manipulação destes arquivos.


\subsubsection{OOHandler}

Inicialmente nomeado em função do OpenOffice.Org, este \textit{handler} é uma implementação específica de comunicação com o LibreOffice, que trata de requisições especificas aos documentos a fim de manipulá-los.


\subsubsection{PDFHandler}

Por utilizar de duas ferramentas, o Poppler e PDFTk, esta classe foi nomeada em função do tipo de arquivo que é responsável, ou seja, arquivos PDF. 


\subsubsection{IMAGEMAGICKHandler}

No que se trata de ferramentas para imagens o ImageMagick é uma das melhores disponível a nível de comando, ela consegue inclusive manipular dados do tipo \textit{Exif}, que são ``metadados'' referentes a imagens.

Assim com base na ferramenta que utiliza, esta classe é responsável pela conversão e extração de ``metadados'' de arquivos de imagem, deseja-se ainda implementar uma funcionalidade de inserção destes, no entanto esta funcionalidade requer uma base de estudos maior.


\subsubsection{FFMPEGHandler}

O FFMPEG  é capaz de manipular arquivos de áudio e vídeo facilitando assim a criação de uma classe que pudesse ser responsável simultaneamente pela manipulação de ambos.

Assim como as demais classes que implementam o IHandler, esta classe é capaz de manipular os ``metadados'' desses arquivos, bem como convertê-los para determinadas extensões do mesmo tipo.


\subsection{IMonitor}

Esta interface foi desenvolvida principalmente com base nos erros anteriormente obtidos com o uso do LibreOffice, no entanto é importante para o sistema como um todo.

Seu uso estabelece controles sobre princípios básicos do sistema, como uso de memória, tempo de requisições, tempo de uso do processo, entre outros.


\subsubsection{Monitor}

Basicamente a classe Monitor funciona como uma simples implementação da interface IMonitor a fim de estabelecer seus atributos principais, como por exemplo o tempo mínimo entre a monitorações do sistema. 

As próximas subseções explicam detalhadamente sobre estes controles através da herança desta classe.


\subsubsection{MonitorMemory}
\label{monitormem}

Nas configurações do CloudOoo existem definições que podem ser modificadas de acordo com o sistema em que vai ser instalado, entre elas existe uma variável responsável pelo uso máximo de memória pelo LibreOffice.

A partir dessa definição, dada em \textit{megabytes}, essa classe monitora o uso da memória do sistema, assim caso esta atinja seu limite máximo de uso, a aplicação é reinicia com intuito de limpar da memória mensagens trocadas e que não foram liberadas de uso da mesma, evitando assim o evento chamado \textit{memory leak}, o qual consiste no uso de toda memória do sistema.


\subsubsection{MonitorTimeout}
\label{monitortim}

Também entre as definições citadas na subseção \ref{monitormem}, existe uma variável responsável pelo tempo limite de execução de um determinado processo, caso este tempo seja excedido ocorre o que chamamos de \textit{timeout} e a aplicação é forçada a parar, sendo reiniciada posteriormente.

A utilidade desta limitação é dada pela idéia de que caso este tempo tenha excedido ocorreu algum erro durante o processo, provavelmente em função da resposta da aplicação, assim reiniciá-la pode resolvê-lo.


\subsubsection{MonitorSleepingTime}

Com intuito de poupar uso do sistema em momentos desnecessários esta classe foi criada para observar o momentos de inutilização da aplicação e após determinado tempo parar a mesma. Assim é possível economizar no uso de recursos, e disponibilizá-los para outras aplicações que possam vir a utilizar o mesmo.


\subsubsection{MonitorRequest}

A fim de conservar a estabilidade do CloudOoo, esta classe implementa um controle em função do valor máximo de requisições que podem ser respondidas por cada instância da aplicação, bem como nas subseções \ref{monitormem} e \ref{monitortim} a variável de requisições limite é estabelecida nas configurações.

Caso o valor informado seja excedido a instância é encerrada, em seguido uma nova instância da mesma aplicação é iniciada.


\subsection{IMimemaper}

Em casos de aplicações como o LibreOffice que representam uma suíte de menores utilitários é preciso reconhecer a extensão de arquivo específica para cada utilitário. 

De forma geral estas extensões são explicitas no nome do arquivo. Entretanto, caso de que esta não sejam explicitas, é preciso reconhecer o tipo de arquivo de alguma outra forma. 

Neste caso existem o \textit{mimetypes}, que são identificadores presentes no conteúdo do arquivo, que permitem decidir sua extensão. 

Esta interface propõe métodos ara lidar com a identificação desses \textit{mimetypes}.

Seu exemplo de uso é a classe Mimemapper a qual será apresentada na próxima subseção.


\subsubsection{Mimemapper}
\label{mimemapper}

O CloudOoo possui seus próprio filtros para identificar e renderizar arquivos dentro de sua própria instância, seção \ref{ihandler}.

No entanto, dada a necessidade de suas aplicações internas, tornou-se necessário identificá-los de forma a torná-los igualmente reconhecível dentro de cada aplicação específica.

No caso do LibreOffice é possível, através do uso do UNO, extrair os \textit{mimetypes} e demais informações como filtros extraídos dos próprios arquivos.


\subsection{IFilter}

Citados na seção anterior, \ref{mimemapper}, os filtros podem ter demais propriedades que ao serem requisitadas precisam estar disponível de forma facilmente utilizável.

Com base neste principio de utilização, esta interface propõe um contrato para trabalhar com os filtros mais complexos da melhor forma possível e igualmente da forma mais simples.


\subsubsection{Filter}

Se comparada as outras classes e suas devidas interfaces, a classe Filter pode ser considerada a que representa o métodos mais simples.

Ao ser iniciada ela guarda todos os dados dos filtro em diversos atributos que foram selecionados prevendo seu uso posterior na aplicação.


\subsection{ILockable}

Quando se possui-se um recurso compartilhado, isto é, um recurso representado por outra aplicação rodando em paralelo à aplicação principal, admiti-se também existir uma região crítica.

Esta visão ocorre devido ao fato que determinadas aplicações não conseguem atender a mais de um processo simultaneamente, assim é necessário controlar essa região crítica prevendo travá-la caso um processo já esteja em usando a mesma.

Com este intuito a interface ILockable foi implementada.


\subsubsection{OpenOffice}

A classe OpenOffice foi exclusivamente criada visando controlar a aplicação LibreOffice, anteriormente conhecida por OpenOffice.Org, a qual foi responsável por maior parte dos erros respondidos do CloudOoo. 

Esta classe estende o uso da interface IApplication, através da classe descrita na seção \ref{application} e ainda implementa a interface anterior, ILockable, para que assim seja possível controlar a aplicação, trancá-la e destrancá-la durante seu processo de uso.

A partir do método \textit{isLocked} é possível verificar se esta aplicação está trancada ou disponível para uso, evitando assim erros, como o \textit{deadlock} por exemplo.


\subsection{ITableGranulator, IImageGranulator, ITextGranulator}

Uma das principais funções desenvolvidas para documentos foi a ``granularização'', ela trata da extração de partes importantes de documentos que não sejam especificamente textos, como por exemplo tabelas e imagens. 

Dada a complexidade dessa tarefa foi necessário a implantação das novas interfaces, para que estes processos fossem realizados.

A interface ITableGranulator é a interface responsável pelo processo de ``granularização'' específico de tabelas presentes nos documentos. Ela implementa funções respectivas a uma tabela comum, baseada nas linhas e colunas dessas tabelas.

Na interface IImageGranulator ocorre a ``granularização'' das imagens presentes nestes documentos, que são extraídas em seu formato original, ou em \textit{PNG}.

Por fim através da interface ITextGranulator é possível partir o documento em textos menores dividindo o mesmo a partir de seus parágrafos, ou mesmo em capítulos.

Apesar dessas funcionalidades tratarem diretamente da ``granularização'' de documentos, podemos subdividi-las em dois tipos de documentos: os documentos livres de formato ODF, onde o processo ocorre a partir estudos realizados anteriormente dos mesmos; e documentos PDF, que são igualmente livres, porém tem um tratamento complexo dada sua falta de detalhamento e especificações.


\subsubsection{OOGranulator}

Esta classe foi desenvolvida especialmente para tratar do documentos ODF.

Suas funcionalidades são escritas com base nos \textit{namespaces} encontrados por padrão no XML que compõe esses documentos.

Até o momento essa classe é a única a implementar as exatas três interfaces da subseção anterior.


\subsubsection{PDFGranulator}

Esta classe implementa apenas as interfaces ITableGranulator e IImageGranulator.

É específica para o tratamento de documentos PDF, que exigem a utilização de bibliotecas e aplicações para ``desmembrá-lo'' e tratar o resultado desta tarefa.

Por não possuir tantas especificações quanto documentos ODF, tratá-lo para ``granularização'' tem sido um trabalho mais complicado e com resultados não tão desejáveis, esta classe se encontra em desenvolvimento e aprimoramento.


\subsection{IManager}

A IManager trabalha como a interface entre o cliente e servidor a fim de estabelecer um protocolo entre ambos.

Ela detém um padrão genérico para troca de informações entre diferentes tipos de arquivos, inclusive vídeos. 

E possui ainda os principais métodos para funcionalidades do CloudOoo, no que desrespeito a todos seus \textit{handlers}, ou seja, módulos para tratar diversos tipos de arquivos.


\subsection{IERP5Compability}

Esta interface estabelece as funcionalidades entre cliente e servidor especificamente para o uso da aplicação ERP5. 

Ela reescreve a chamada dos métodos da aplicação OOOD para utilizarem os novos métodos sem interferir nas requisições feitas pelo uso do ERP5, e outros possíveis clientes que utilizassem a antiga plataforma.


\subsubsection{Manager}

A classe Manager implementa as classes IManager, IERP5Compability, ITableGranulator, ITextGranulator e IImageGranulator com o propósito de interligar suas funcionalidades ao cliente que venha a requeri-las, em outras palavras esta classe é a principal responsável pela conectividade entre cliente, servidor, aplicação e funcionalidades.

Nela são absorvidos os dados importantes para o funcionamento do CloudOoo, como por exemplo a base de \textit{mimetypes}, os \textit{handlers} disponíveis neste, as pastas de trabalho deste, entre outros dados.

Assim a partir desta classe é possível iniciar o servidor do CloudOoo.
\subsection{Tabela de Novas funcionalidades}

Na tabela \ref{funclooo} estão representadas algumas das funcionalidades do CloudOoo em função do que foi desenvolvido neste trabalho:

\begin{table}
  \caption{Comparação de funcionalidades do CloudOoo.}
  \label{funclooo}
  \begin{tabular}[b]{|p{3.5cm}|c|c|c|p{1.5cm}|}
  \hline
  Funcionalidades & OOOD 1.0 & OOOD 2.0 & CloudOoo 1.24 & Neste trabalho \\
  \hline
  Conversão de documentos & X & X & X & - \\
  \hline
  Manipulação de metadados dos documentos & - & X & X & - \\
  \hline
  ``Granularização'' de documentos & - & - & X & - \\
  \hline
  ``Granularização'' de documentos PDF & - & - & X & X \\
  \hline
  Controle de problemas com sistema & - & X & X & X \\
  \hline
  Controle de problemas com LibreOffice & - & X & X & - \\
  \hline
  Conversão de PDF & - & - & X & X \\
  \hline
  Conversão de Imagens & - & - & X & - \\
  \hline
  Conversão de Áudio & - & - & X & X \\
  \hline
  Conversão de Vídeos & - & - & X & X \\
  \hline
  Manipulação de metadados áudio e vídeo & - & - & X & X \\
  \hline
  \end{tabular}
\end{table}

