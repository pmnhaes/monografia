\chapter{TECNOLOGIAS SIMILARES}
\thispagestyle{empty}

Este capítulo visa apresentar um estudo a respeito de outras aplicações que se assemelham ao tipo de aplicação que representa o CloudOoo, isto é, aplicações web capazes de realizar conversões de documentos e arquivos de multimídia, e que ainda pudessem manipular informações destes arquivos.

Este estudo foi baseado especialmente em aplicações livres e de código aberto que se encontram disponíveis para uso e colaboração tal como o CloudOoo, dado o pequeno número dessas aplicações também foram estudadas páginas que ofereciam o serviço de conversão on-line.

Entretanto cabe observar que dentre todas aplicações estudados não foi encontrada em nenhuma delas a funcionalidade de extração ou inserção de dados disponíveis no CloudOoo.


\section{InOut}

O InOut é o serviço web que mais se aproximou de ser uma tecnologia similar ao CloudOoo. Ele foi desenvolvido exclusivamente para lidar com arquivos e suas conversões.

Este serviço é capaz de converter documentos dos formatos:

\begin{itemize}
    \item{.pdf: Adobe PDF;}
    \item{.doc, .docx: Microsoft World 2003/2007/2010;}
    \item{.odt: Open Document Textfile;}
    \item{.sxw: Open Office Writer;}
    \item{.rtf: Rich Text Format;}
    \item{.ppt, .pttx: Microsoft Powerpoint 2003/2007/2010;}
    \item{.odp: Open Document Presentation;}
    \item{.odc: Open Document Chart Gradfic;}
    \item{.odi: Open Document Image;}
    \item{.sxi: Open Office Impress;}
    \item{.xls, .xlsx: Microsoft Excel 2003/2007/2010;}
    \item{.ods: Open Document Spreadsheeet;}
    \item{.sxc: Open Office 1.0 Spreadsheet;}
    \item{.csv: Comma separated file.}
\end{itemize}


Para os formatos de:
\begin{itemize}
    \item{.pdf: Adobe PDF;}
    \item{.doc, .docx: Microsoft World 2003/2007/2010;}
    \item{.odt: Open Document Textfile;}
    \item{.sxw: Open Office Writer;}
    \item{.rtf: Rich Text Format.}
\end{itemize}


Vídeos dos formatos:
\begin{itemize}
    \item{.mpeg: MPEG videos;}
    \item{.mp4: MPEG 4 format;}
    \item{.qt: Quicktime videos;}
    \item{.avi: Microsoft AVI videos;}
    \item{.wmv: Windows Media Video 7 and 8;}
    \item{.asf: Microsoft Advanced Streaming Format;}
    \item{.ogv: Ogg Theora videos;}
    \item{.flv: Flash videos;}
    \item{.f4v: MPEG 4 encoded Flash videos;}
    \item{.mkv: Matroska videos;}
    \item{.m2ts: AVCHD or Blue-Ray video tracks;}
    \item{.vob: MPEG encoded video tracks from DVDs;}
    \item{.fli: FLIC videos;}
    \item{.movie: SGI-Movie videos;}
    \item{.dl, .gl: DL Animation format.}
\end{itemize}


Para os formatos:
\begin{itemize}
    \item{.mpeg: MPEG videos;}
    \item{.mp4: MPEG 4 format;}
    \item{.flv: Flash Videos.}
\end{itemize}


De formatos de áudio:
\begin{itemize}
    \item{.wav: Wave audio files;}
    \item{.mp3: MP3 audio files;}
    \item{.mp2: MPEG audio files;}
    \item{.midi: MIDI audio files;}
    \item{.snd: Raw audio files;}
    \item{.wma: Windows Media audio files;}
    \item{.m4a: MPEG 4 audio files;}
    \item{.aac: MPEG 4 audiostream;}
    \item{.ac3: Dolby Digital audio files;}
    \item{.aif: Apple Audio Interchange files;}
    \item{.flac: Free Lossless audio files;}
    \item{.ra: Real audio with the extensions *.ra and *.ram;}
    \item{.ogg: Ogg Vorbis audio files;}
    \item{.mka: Matroska audio files.}
\end{itemize}

Para os formatos:
\begin{itemize}
    \item{.wav: Wave audio files;}
    \item{.mp3: MP3 audio files.}
\end{itemize}


Além de imagens dos formatos:
\begin{itemize}
    \item{.jpeg: JPEG graphics;}
    \item{.gif: GIF graphics;}
    \item{.png: PNG graphics;}
    \item{.tiff: TIFF graphics;}
    \item{.bmp: Windows Bitmap graphics;}
    \item{.wmf: Windows Metafile graphics;}
    \item{.ai: Adobe Illustrator vector graphics;}
    \item{.eps: Encapsulated post script vector graphics;}
    \item{.ps: Post script files;}
    \item{.psd: Adobe Photoshop graphics;}
    \item{.tga: Targa graphics;}
    \item{.pcx: Zsoft paintbrush graphics;}
    \item{.pct: Mac Pict graphics.}
\end{itemize}

Para os formatos:
\begin{itemize}
    \item{.jpg: JPEG graphics;}
    \item{.png: PNG graphics;}
    \item{.gif: GIF graphics.}
\end{itemize}

Através de uma breve analise às listas de conversões é possível notar que embora reconheça muitos padrões o InOut preferência a saída de arquivos em formatos livres de suas conversões.

Para comunicação entre seu serviço e qualquer aplicação que queira requisitá-lo, existe uma extensa documentação na página \url{https://api.inout.io/InOut - API Documentation.html}, esta documentação é especialmente desenvolvida para outros desenvolvedores.

Nessa documentação é recomendado o uso de RESTful e JSON para realizar a comunicação com o serviço, entretanto a linguagem a se utilizar fica por conta de cliente.

No código \ref{inout} é possível notar um exemplo de uso do InOut em que é requerido todos os padrão de conversão para seu uso através do JSON:

{\singlespace
\begin{lstlisting}[caption=Requisição de perfis do InOut,language=bash,label={inout}]

$ curl -u key:secret https://api.inout.io/profiles/


{"ok": true, "result": [
   {"id": "IMAGE_TO_JPEG"},
   {"id": "IMAGE_TO_PNG"},
   {"id": "IMAGE_TO_GIF"},
   // ...
]}
\end{lstlisting}
}

Demais exemplos de uso do InOut podem ser encontrados na página da documentação citada anteriormente.


\section{ServPDF}

É um serviço web capaz de converter documentos, sejam eles de formatos abertos ou proprietários pela Microsoft.

Ele permite que qualquer documento possa ser convertido para PDF, e que qualquer PDF possa ser convertido em Postscript ou imagens.

É uma ferramenta extremamente simples baseada no Microsoft Office e modificada para atender ao LibreOffice.


\section{Conversões Online}

Fora esses serviços a gama de projetos livres para conversão de arquivos e documentos em geral é extremamente escassa. Existem muitas versões \textit{desktop} e também versões de serviço on-line, neste caso os serviços são hospedados em páginas que ficam disponíveis gratuitamente para usuários.

A desvantagem desses serviços em relação ao CloudOoo é de que normalmente seus arquivos são limitados a um tamanho específico pelos proprietários da página, enquanto para um usuário que hospeda um CloudOoo é possível modificar esse limite de acordo com a máquina qual o mesmo será instalado, podendo este limite ser bem maior, entretanto dependendo do usuário para instalá-lo.

Assim é possível concluir que para um usuário final que não possua conhecimentos avançados de serviços web pode ser bem mais agradável utilizar essas páginas. As próximas subseções mostram alguns exemplos interessantes para seu uso.

\subsection{You Convert It}

O You Convert It é uma das páginas disponíveis na web para conversão de documentos, imagens, vídeos e áudio on-line. É uma das páginas mais aconselháveis pois possui o limite de arquivos no tamanha de até 1GB.

\subsection{Convert Files}

Assim como o You Convert It, o Convert.Files é capaz de converter desde documentos a arquivos de multimédia, entretanto com um limite bem menor de apenas 150MB por arquivo. Nele existe a vantagem de que o usuário possa mandar os arquivos convertidos para seus celulares ou aparelhos moveis em geral.

\subsection{Zamzar}

Similar aos exemplos anteriores o Zamzar é capaz de realizar a conversão entre qualquer arquivo, entretanto seu limite tente função de arquivos de 100MB e até 5 conversões diárias, apesar disso é um dos serviços mais populares para conversão on-line.

