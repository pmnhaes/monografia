\chapter{INTRODUÇÃO}

Com a evolução constante da internet, que hoje é considerada um universo crescente de paginas, aplicações, vídeos, imagens e outros arquivos amplamente disponíveis e interativos, os serviços web provaram ser uma forma rápida e consideravelmente simples de comunicação através destas paginas e aplicações nelas existentes.
Por isso, atualmente, acompanhamos uma migração não só de tarefas antiquadas de ampla exigência manual, bem como de aplicações que são migradas do nível de \textit{desktop} para uma rede de nuvens.
O \textit{cloud computing} conhecido como computação nas nuvens consiste em permitir a  qualquer usuário que acesse uma determinada aplicação independente da plataforma em que o mesmo se encontre, esta nova forma de interação entre maquina e usuário se torna mais ampla a medida que o avanço tecnológico nos permite maior e mais rápido acesso a internet \cite{Alecrim 2008}.
Seguindo esta tendência o Núcleo de Pesquisa em Sistemas de Informação(NSI) utilizavam uma ferramenta chamada OpenOffice.org Daemon, desenvolvida originalmente pela Nexedi SA, a qual tinha por objetivo converter documentos de escritório. No entanto a partir do uso prolongado desta ferramenta , erros  sérios  eram apresentados, tais como perda de conexão ou \textit{deadlock} no OpenOffice.org utilizado pela aplicação, entre outros. Assim notou-se a necessidade de corrigir a ferramenta a fim de obter certa estabilidade, esta iniciativa partiu de uma ação conjunta do NSI com a Nexedi SA , porém após realizada a analise de mesma, provou-se inviável corrigi-la em função de novas demandas percebidas por ambas. 
Como nova meta foi proposta a criação de uma ferramenta que fosse capaz de manter as conversões  anteriores, sem que fossem detectados os mesmo erros anteriores, e que ainda fosse capaz de extrair e manipular informações pertencentes aos documentos em questão. 
Em 2010, a ferramenta Web Service OOOD 2.0 foi lançada. Seguindo em parte o nome da antecessora, esta correspondia as expectativas implícitas na ultima analise realizada. Em função dos objetivos alcançados, e com base no aprendizado envolvido no desenvolvimento desta, novas metas foram traçadas, entre elas que a ferramenta fosse capaz de possibilitar a mesma manipulação de documentos para outros tipo de arquivos.

\section{Objetivo}
Este trabalho tem como principal objetivo apresentar sucessora do OOOD 2.0, com intuito de demonstrar suas atualizações e ampliações que vieram a resultar na nova ferramenta conhecida por CloudOoo, a qual provou-se capaz de manipular outros formatos de arquivos, ainda que prematuramente ainda não tenha mesma estabilidade, mas que caminha para isto.

\section{Estrutura do trabalho}
O primeiro capitulo deste trabalho explica um pouco sobre as principais ferramentas que permitiram ao CloudOoo alcançar os objetivos inicialmente traçados, demonstrando um pouco sobre cada uma delas.
No segundo capitulo tem-se uma apresentação da ferramenta Web Service OOOD 2.0, demonstrando um pouco das soluções desenvolvidas para os problemas da ferramenta OOOD 1.0, e suas inovações.
Ao terceiro capitulo viso apresentar como as novas modificações destas ferramentas que vieram a resultar no CloudOoo, que atualmente é capaz de manipular diversos de arquivos em larga escala. 
No quarto capitulo apresento um breve estudo de caso do desenvolvimento da mesma seguindo com um estudo anterior onde esta foi utilizada para demonstrar sobre desenvolvimento de ferramentas usando \textit{agile}.
Por fim no quinto capítulo são apresentadas as conclusões sobre o trabalho com a mesma e de proposta futuras para a melhoria desta.
