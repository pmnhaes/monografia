\chapter{INTRODUÇÃO}
\thispagestyle{empty}

Segundo \cite{TESLA}, a Internet é um produto descendente de um experimento militar americano que tentava formar uma rede que não fosse vulnerável ao ataque inimigo, e que desde seu processo de desmilitarização expandiu ao redor mundo, mesmo em continentes afastados e pouco populosos como a Antártica.  

Ela foi fundamental para criação do \textit{Cloud computing}, em português computação nas nuvens, que consiste em permitir ao usuário acesso a diversas aplicações e ao máximo de funcionalidades que esta nova forma de interação entre maquina e usuário possa disponibilizar, independente da plataforma em que o mesmo se encontre, tornando-se cada vez mais ampla a medida que o avanço tecnológico permite maior acesso a Internet.

Seguindo esta tendência, mesmo antes de sua popularização, o Núcleo de Pesquisa em Sistemas de Informação(NSI) já trabalha anos no desenvolvimento e melhoria do projeto Biblioteca Digital da RENAPI, o qual visa disponibilizar um acervo bibliográfico digital para disseminação de científico tecnológico produzido na rede de Instituições de Educação Profissional Científica e Tecnológica (EPCT).

Assim o NSI passou a utilizar, entre outras, a ferramenta \textit{OpenOffice.org Daemon}, a qual foi desenvolvida originalmente pela empresa francesa Nexedi SA. Essa ferramenta possuía por objetivo a conversão de documentos.

No entanto a partir do uso prolongado da mesma, foram identificados erros considerados em parte graves para uma aplicação desta extensão. Entre esses é possível citar perda de conexão, \textit{deadlock} no OpenOffice.org utilizado pela aplicação, estouro de memória entre outros.

Assim dada a necessidade de corrigir estes erros a fim de obter estabilidade, através da parceria do NSI com a Nexedi,foi realizada a analise desta ferramenta.

Entretanto a análise provou que era inviável corrigir tal ferramenta em função de novas demandas também percebidas.

Definiu-se assim uma nova meta, a proposta da criação de uma ferramenta que fosse capaz de manter as conversões desta, mas que tivesse seus principais erros corrigidos, e que ainda fosse capaz de extrair e manipular informações pertencentes aos documentos em questão.

Em 2010, a ferramenta Web Service OOOD 2.0 foi apresentada. Seguindo em parte a sigla da antecessora, esta correspondia as expectativas implícitas na ultima analise realizada, no entanto, dado que os objetivos anteriores foram alcançados, e com base no aprendizado envolvido neste desenvolvimento, novas metas foram traçadas.

Além de poder manipular documentos tornou-se desejável também que a ferramenta fosse capaz de manipular arquivos de imagem, vídeo, áudio e PDF.

\section{Objetivo}

Este trabalho tem como principal objetivo apresentar a ferramenta sucessora ao OOOD 2.0, com intuito de demonstrar suas atualizações e ampliações. O CloudOoo provou-se capaz de manipular outros formatos de arquivos, ainda que prematuramente não tenha mesma estabilidade, mas que caminha para isto.

\section{Estrutura do trabalho}

Este trabaho se divide em cinco capítulos a partir deste primeiro:

O segundo capítulo deste trabalho cita e explica sobre as principais ferramentas que permitiram ao CloudOoo alcançar os objetivos inicialmente traçados, e que estão diretamente envolvidas a este, demonstrando um pouco sobre cada uma delas.

No terceiro capítulo é apresentada a nova estrutura sobre a qual o CloudOoo vem sendo desenvolvido, e sobre sua necessidade para seu funcionament.

No quarto capítulo é apresentado um breve estudo de caso do desenvolvimento desta ferramenta, avaliando seu desenvolvimento; falando sobre aplicações que o utilizam e demonstrando sobre sua atual estabilidade com base em estudos anteriores.

Por fim no quinto capítulo são apresentadas as conclusões sobre este trabalho e de proposta futuras para a melhoria desta ferramenta.
